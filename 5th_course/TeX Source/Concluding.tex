\newpage
\parindent=1cm %красная строка
\addcontentsline{toc}{section}{Заключение} %Убираем номер , даём имя в оглавлении 
\section*{Заключение} %сам текст заголовка 

Разработанный и протестированный алгоритм имеет как преимущества, так и недостатки, рассмотрим их подробнее.


Преимущества: \\
\begin{itemize}
	\item Полная симуляция одновременно происходящих параллельных процессов;
	\item Минимальное, по сравнению с другими рассмотренными алгоритмами, число итераций;
	\item Возможность экспортировать данные в табличные процессоры для пост-обработки;
	\item Низкие затраты по памяти -- запущенное приложение при 10 активных ракетах-целях, 30 активных  ракетах-перехватчиках и 100 ракетах-перехватчиков на складе потребляет около 500 Мб ОЗУ;
	\item Высокие возможности распараллеливания алгоритма на критических участках -- потенциальный источник еще большей оптимизации исполнения алгоритма.
\end{itemize}

Недостатки: \\
\begin{itemize}
	\item Не самая высокая скорость сходимости среди рассмотренных;
	\item Симуляция параллельных событий требует процессора со значительным числом ядер и логических потоков, которые часто отсутствуют на встраиваемых системах, реально используемых в подобной технике. Так, описанное выше кол-во объектов создает около 50 процессов и более 100 потоков;
	\item Алгоритм рассматривает упрощенную схему полёта ракеты-цели и не учитывает возможных положений, допускающих оптимизацию процесса перехвата.
\end{itemize}

Были решены следующие задачи:

\begin{itemize}
	\item Описана СПРО для которой  построена модель и агенты, участвующие в ней;
	\item Описана система принятия решений и накладываемые на нее ограничения, алгоритм принятия решений;
	\item алгоритм принятия решений был реализован в виде ПО, протестирован и сравнён с другими подобными алгоритмами;
	\item были решены различные задачи, возникающие в области многоточечного программирования;
	\item были сделаны выводы о потенциале и возможностях развития развития данного алгоритма.
\end{itemize}

К сожалению, не удалось реализовать дополнительную цель и построить полноценный интерфейс приложения, позволяющий ввод-вывод данных и, самое важное, демонстрацию работы алгоритма посредством показа назначения ракет-перехватчиков на цели и непосредственно перехватом.

Несмотря на неидеальные параметры работы алгоритма, он всё же демонстрирует высокую скорость исполнения. Отметим, что как сам алгоритм АРЧПОСОО, так и модель, имитирующая СПРО, допускают в будущем модификации как расширяющие функционал СПРО, так и модифицирующие непосредственно работу самого алгоритма АРЧПОСОО.

