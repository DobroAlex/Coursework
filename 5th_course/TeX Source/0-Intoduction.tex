\newpage
\parindent=1cm %красная строка?
\begin{center}
	\addcontentsline{toc}{section}{Введение} %Убираем номер , даём имя в оглавлении 
	\section*{Введение} %сам текст заголовка 
	\pagestyle{plain} % нумерация выкл.
	\setcounter{page}{3} % начать нумерацию с номера три
\end{center}


Системы противоракетной обороны играют важную роль в обеспечении защиты государства от баллистических ракет и позволяют поддерживает стратегический паритет. Математическое моделирование таких систем представляет собой сравнительную дешевую в терминах людских и материальных ресурсов и безопасную для окружающей среды и людей альтернативу проведению реальных испытаний. Как правило, системы противоракетной обороны (далее ПРО) представляют собой компл\'eксную систему, имеющую нелинейный характер и состоящую из множества подсистем реального времени, работающих параллельно. Необходимость точного описания порядка взаимодействия компонентов системы и характера этих взаимодействий и их результата является ключевым важным условием оценки эффективности всей системы ПРО.

Следует заметить, что многие современные системы являются многослойными, то есть содержат не просто отдельные компоненты, выполняющие действия на отдельном этапе, а целые подсистемы, выполняющие эти действия.

Область данной работы не является принципиально новой и неисследованной и уже многие авторы внесли свой вклад в разработку построения максимально точных моделей, способных описать систему ПРО. Большинство работ, связанных с данной темой использует один из следующих подходов: традиционное детерминированное моделирование с использованием систему дифференциальных уравнений, инженерный подход к описанию системы, вычислительный эксперимент. Однако, упомянутые выше подходы имеют ряд недостатков и не могут описать внутреннюю сложность систем и связей между ними. В противоположность этим классическим методам, мультиагентное моделирование предоставляет подход к моделированию по принципу <<сверху вниз>>, т.е. заменяет сложный подход описания всей системы с помощью одной системы уравнений описанием компонентов и связей между ними, что и формирует полное описание системы, что в свою очередь является более подходящим подходом при моделировании компл\'eксных систем.

Этот интересный и новый метод не мог остаться незамеченным и уже рассмотрен в ряде работ: 
%TODO: список работ со стр 514
Все эти работы нацелены на решение задач  управления, но не решают задачу автономного принятия решения. Решение задачи принятия решения в данных условиях особенно интересно, т.к система ПРО представляет собой систему реального времени, и особенно важным ограничением является ограничение по времени на принятие решения -- обычно, полный жизненный цикл баллистической ракеты составляет около 28 минут. Данное ограничение по времени накладывает условие на временн\'yю сложность алгоритма принятия решения -- он должен завершиться и завершиться намного раньше, чем это временное ограничение, т.к. помимо принятия решения необходимо также провести пуск ракеты-перехватчика и дождаться поражения или не поражения ей цели.

Таким образом, ожидается, что построение системы принятия решения улучшит характеристики моделируемой системы, ровно как и оптимизация этой системы принятия решений еще боле увеличит эффективность системы в целом. Ли %TODO: [21] из работы
в своей работе уже предложил использование модифицированного алгоритма метода роя частиц.

Целью данной работы является реализация алгоритма принятия решений для системы ПВО, предназначенной для перехвата межконтинентальных баллистических ракет, и тестирование этой системы на имитационном ПО. Для этого необходимо решить следующий комплекс задач:

\begin{itemize}
	\item Описать систему противоракетной обороны для которой будет строиться модель;
	\item описать агентов;
	\item интегрировать описанных агентов в модель системы ПРО;
	\item описать систему принятия решений и накладываемые на нее ограничения;
	\item описать алгоритм принятия решений и провести его оптимизацию;
	\item провести тестирование полученного алгоритма в имитационном ПО;
	\item проанализировать результаты и сделать выводы. 
\end{itemize}

Для решения поставленного комплекса задач использовались методы математической статистики и теории вероятности, дискретной математики, математического анализа и теоретической механики. Разработанная модель основывается на методах агентного / мультиагентного моделирования.

Объект исследования: изучение алгоритмов оптимизации методом роя частиц.

Предмет исследования: построение и анализ модели ПРО, имеющей блок принятия решений, основанный на оптимизации методом роя частиц.

Практическая ценность работы: полученный алгоритм может быть использован как база для разработки более точных алгоритмов, специфичных для конкретных средств перехвата и их целей. Данный алгоритм способен имитировать перехват целей с заданными пространственно-скоростными ограничениями средствами с аналогичными ограничениями и описывает систему ПРО с фиксированным количеством компонентов. 



 \newpage %перевод следующей главы на новую страницу с изменеием номера страницы в оглавлении 
