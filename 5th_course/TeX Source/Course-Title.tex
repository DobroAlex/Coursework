    \begin{titlepage}
    \newpage
	\pagestyle{empty} % нумерация выкл.
    \begin{center}
    
	
	{\fontsize{13}{15.6}\selectfont МИНИСТЕРСТВО ОБРАЗОВАНИЯ И НАУКИ РОССИЙСКОЙ ФЕДЕРАЦИИ}\\ 
    \normalsize  {Федеральное государственное автономное образовательное учреждение высшего образования} \\
    
    \large \textbf{<<Крымский  федеральный  университет имени В. И. Вернадского>>} \\  \vspace{2mm}
    (ФГАОУ ВО «КФУ им. В. И. Вернадского»)\\
    
    \textbf{Таврическая академия (структурное подразделение ) \\
    \vspace{2mm}
    Факультет математики и информатики} \\
    \vspace{2mm}
    Кафедра прикладной математики 
    \end{center}
    \vspace{1em}

    \begin{center}
	\normalsize Консманов Алексей Витальевич \\
    \LARGE \textbf{Использование мультиагентного подхода в военном деле} \\
    \vspace{1em}
    \normalsize Курсовая работа 
    \end{center}

    \vspace{1em}
    Обучающегося \hspace*{3cm} \underline{1	} курса 
    
    
    Направления подготовки\hspace*{8mm} \underline{01.04.04. Прикладная математика}
    
        
    Форма обучения\hspace*{2.75cm} \underline{очная}\\
    
    
    Научный руководитель
    
    доцент кафедры прикладной  
    
    математики, к. ф.-м. н. \hspace*{5cm} Ю.Ю. Дюличева
    %\begin{center}
    %	\begin{tabbing}	%http://www.intuit.ru/studies/courses/1137/137/lecture/3835%3Fpage%3D5
    %		\hspace{3cm}Обучающегося \hspace{3cm} \textbf{3 курса}\\ %Быдлокод?
    %		\hspace{2.7cm}Направления подготовки \hspace*{5mm}  \textbf{01.03.04}\\
    %		\hspace{3cm}Форма обучения \hspace{26mm} \textbf{очная}
    %	\end{tabbing}
    
%	\vspace {3em}
%    \flushleft Научный руководитель \hspace{20mm}  старший преподаватель 
    
%    \hspace{75mm}кафедры прикладной математики  
    
    
%    \hspace{75mm}В. А. Лушников
%	\end{center}
    \vspace{\fill}

    \begin{center}
    Симферополь 2020
    \end{center}

    \end{titlepage}
