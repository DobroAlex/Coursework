\parindent=1cm %красная строка

\begin{center}
		
		\section{Мультиагентное моделирование системы поддержки принятия решений в составе системы противоракетной обороны}
		
\end{center}

\subsection{Описание системы ПРО} 

Как правило, процесс противоракетной обороны (далее ПРО) состоит из последовательности этапов, включающих ранее обнаружение, отслеживание, распознавание, принятие решения и непосредственный перехват цели, достигаемый посредством взаимодействия всех компонентов системы ПРО (далее СПРО). 

Не вдаваясь в технические подробности и устройство составных компонентов СПРО, отметим только сами компоненты и функции, выполняемые этими компонентами:

\begin{itemize}
	\item Радар раннего обнаружения фиксирует факт запуска ракет и получает приблизительные данные о положении и скорости;
	\item отслеживающий радар <<ведёт>> ракету-перехватчик до завершения перехвата или промаха, получая инструкции из командного центра и передавая их ракете-перехватчику;
	\item командный центр выполняет роль коммуникационного звена и центра обработки информации; производит оценку траектории ракеты, на основании  данных от радара раннего обнаружения; отправляет информацию отслеживающему радару;  создание (генерация) плана перехвата ракеты и отправка его ракете-перехватчику в подходящий момент времени;
	\item ракета-перехватчик выполняет задачу перехвата и способна к ограниченному маневрированию.
\end{itemize} 

Очевидно, что для полного цикла работы СПРО необходима и ракета-цель. 




\subsection{Агентное моделирование и понятие <<агента>>}

\newpage % следующая глава с новой страницы