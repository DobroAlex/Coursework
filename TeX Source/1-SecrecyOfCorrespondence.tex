\parindent=1cm %красная строка? 
\begin{center}
		
		\section{Понятия <<тайна связи>> и <<личная переписка>> в правом и информационном аспектах}
		
\end{center}

\subsection{Понятия в правовом аспекте} 

Так как все пользователь информационных систем являются в первую очередь гражданами правовых государств и объектами и субъектами права, рассмотрение основных понятий начнём с правого аспекта вопроса.

Согласно статье 63 федерального закона <<О связи>>: %LITER: http://home.garant.ru/#/document/186117/paragraph/645:1
<<На территории Российской Федерации гарантируется тайна переписки, телефонных переговоров, почтовых отправлений, телеграфных и иных сообщений, передаваемых по сетям электросвязи и сетям почтовой связи.>> Исходя из данного закона, в дальнейшем под <<\textbf{тайной связи}>>  будет подразумеваться совокупность тайны переписки, телефонных разговоров, почтовых отправлений, телеграфных и иных сообщений, передаваемых по сетям электросвязи и сетям почтовой связи. Также, в дальнейшем к сетям электросвязи будем относить и Интернет. 


 
\newpage % следующая глава с новой строки