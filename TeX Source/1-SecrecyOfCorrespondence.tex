\parindent=1cm %красная строка
\begin{center}
		
		\section{Понятия <<тайна связи>> и <<личная переписка>> в правом и информационном аспектах}
		
\end{center}

\subsection{Понятия в правовом аспекте} 

Так как все пользователь информационных систем являются в первую очередь гражданами правовых государств и объектами и субъектами права, рассмотрение основных понятий начнём с правого аспекта вопроса.

Согласно статье 63 федерального закона <<О связи>>: %LITER: http://home.garant.ru/#/document/186117/paragraph/645:1
<<На территории Российской Федерации гарантируется тайна переписки, телефонных переговоров, почтовых отправлений, телеграфных и иных сообщений, передаваемых по сетям электросвязи и сетям почтовой связи.>> 
Федеральный закон "Об информации, информационных технологиях и о защите информации" от 27.07.2006 \textnumero 149
%LITER: http://www.consultant.ru/cons/cgi/online.cgi?req=doc&base=LAW&n=221952&fld=134&dst=100144,0&rnd=0.40247948281489154#07281395619440368
определяет набор правовых, организационных и технических мер,  целью которых является  защита информации от неправомерного доступа, модификации, блокирования, копирования и распространения. Также вводится ответственность за правонарушения в сфере информационных технологий и защиты информации.  Устанавливается понятие \textbf{информации} как сведений (данных, сообщений) независимо от их формы представления, \textbf{информационно-телекоммуникационной сети } как "технологической системы, предназначенной для передачи по линиям связи информации, доступ к которой осуществляется с использованием средств вычислительной техники ". 

Исходя из данных законов, в дальнейшем под <<\textbf{тайной связи}>>  будет подразумеваться совокупность тайны переписки, телефонных разговоров, почтовых отправлений, телеграфных и иных сообщений, передаваемых по сетям электросвязи, сетям почтовой связи и информационно-телекоммуникационным сетям.  Из определения последней очевидно, что к такой сети  можно отнести сеть <<Интернет>>

Понятие \textbf{<<личная переписка>>} в данной работе  подразумевает информацию личного характера, не составляющую коммерческую, государственную или другую тайну, передаваемую любым способом,  который  используется   в <<тайне связи >> и <<тайне связи в Интернете>>.

Подобные законы существуют в большинстве развитых стран. Например, четвертая поправка  к Конституции США  гласит : <<Право народа на охрану личности, жилища, бумаг и имущества от необоснованных обысков и арестов не должно нарушаться. Ни один ордер не должен выдаваться иначе, как при наличии достаточного основания, подтвержденного присягой или торжественным заявлением; при этом ордер должен содержать подробное описание места, подлежащего обыску, лиц или предметов, подлежащих аресту>>.%LITER: https://photos.state.gov/libraries/adana/30145/publications-other-lang/RUSSIAN.pdf 
%ORIGIN???	
В ЕС  с 2016 года  на смену Data Protection Directive(директива 95/46/ЕС) пришел General Data Protection Regulation, GDPR (Общеевропейский регламент о персональных данных), обязательный для всех организаций на территории ЕС, осуществляющих  обработку персональных данных, в том числе, связанных с тайной связи и переписки.   Подл действия регламента попадают данные, позволяющие непосредственно или косвенно определить личность человека, к которому эти данные относятся: IP-адрес, cookie ID, банковские данные, персональная информация и переписка, имя,адрес электронной почты, проживания или фактического нахождения. Физические лица получат право на забвение, на исправление, доступа -- знать, какая информация хранится и как обрабатывается, на ограниченную обработку -- блокировать или запрещать обработку , перенос данных и возражение -- аналогично праву на блокировку применимо к маркетингу и научным статистическим исследованиям.  
%LITER: http://www.tadviser.ru/index.php/%D0%A1%D1%82%D0%B0%D1%82%D1%8C%D1%8F:%D0%97%D0%B0%D1%89%D0%B8%D1%82%D0%B0_%D0%BF%D0%B5%D1%80%D1%81%D0%BE%D0%BD%D0%B0%D0%BB%D1%8C%D0%BD%D1%8B%D1%85_%D0%B4%D0%B0%D0%BD%D0%BD%D1%8B%D1%85_%D0%B2_%D0%95%D0%B2%D1%80%D0%BE%D1%81%D0%BE%D1%8E%D0%B7%D0%B5_%D0%B8_%D0%A1%D0%A8%D0%90 



Введём понятие <<\textbf{тайны связи в Интернете}>>, дополнив исходное понятие и изменив область приложения. Под <<тайной связи в Интернете>> в дальнейшем будет подразумеваться совокупность правовых норм, алгоритмов и методов сохранения секретности и  непубличности (известность и доступность только абонентам ) содержимого самого сообщения, информации о его абонентах (получателе или получателях и отправителе), условиях передачи сообщения (время, место отправки и получения, используемое при этом оборудование). 

Нарушениями тайны связи не является :
\begin{itemize} %ORIGIN???
		\item Прослушивание (в том числе и обыск) без ордера   в случае проведения контрразведывательных операций. Однако подобное допустимо только при условии наличия достаточных оснований и обоснования того, почему в конкретном случае получение ордера не целесообразно. При этом правоохранительные органы могут искать лишь доказательства, подтверждающие факты действия разведывательных органов иностранных государств.
		\item Во многих странах заключённые и их вещи могут обыскиваться без каких-либо оснований в любое время, так-как подобное является частью режима лишения свободы, применённого к заключённому по решению суда. Аналогичное касается электронной и прочих видов связи.
		\item Контроль почтовых отправлений, телеграфных и иных сообщений, прослушивание телефонных переговоров, снятие информации с технических каналов связи являются видами оперативно-разыскных мероприятий. Их проведение в Российской Федерации  допустимо на основании судебного решения и при наличии информации о  событиях или действиях, создающих угрозу государственной, военной, экономической или экологической безопасности Российской Федерации; о лицах, подготавливающих, совершающих или совершивших противоправное деяние, по которому производство предварительного следствия обязательно; о признаках подготавливаемого, совершаемого или совершенного противоправного деяния, по которому производство предварительного следствия обязательно.
		
\end{itemize}

 Применимо к особенностям организации общения и передачи данных в Интернете, под   \textbf{ <<нарушением тайны связи}>> будут подразумеваться следующие ситуации : 
\begin{itemize}
	\item Передача стороной, предоставляющей услуги связи, данных об абонентах связи, времени связи и прочих параметрах сообщений третьим лицам.
	\item  Проведение атаки на любые  физические компоненты коммуникационных сетей  : ЭВМ пользователей или стороны, предоставляющей услуги связи, сетевое оборудование, серверы; атаки типа <<Man in the middle>>, выполняемые непосредственно на линиях связи.
	\item Использование правоохранительными органами прослушивающего оборудования без соответствующих санкций (ордера) суда или другие действия, выходящие за рамки полномочий правоохранительных органов, ведомств и силовых структур данного государства.
	\item Перехват сообщений на аналоговых носителях с целью их изучения и/или модификации с последующей передачей изначальному адресату; аналогичный перехват с целью изучения и уничтожения  или перехват с целью уничтожения без изучения.
	\item Преступная халатность, повлекшая попадание частных данных в руки третьих лиц.
	 
\end{itemize} 
 
\newpage % следующая глава с новой строки