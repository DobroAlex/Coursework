\parindent=1cm %красная строка? 
\begin{center}
		
		\section{Понятия <<тайна связи>> и <<личная переписка>> в правом и информационном аспектах}
		
\end{center}

\subsection{Понятия в правовом аспекте} 

Так как все пользователь информационных систем являются в первую очередь гражданами правовых государств и объектами и субъектами права, рассмотрение основных понятий начнём с правого аспекта вопроса.

Согласно статье 63 федерального закона <<О связи>>: %LITER: http://home.garant.ru/#/document/186117/paragraph/645:1
<<На территории Российской Федерации гарантируется тайна переписки, телефонных переговоров, почтовых отправлений, телеграфных и иных сообщений, передаваемых по сетям электросвязи и сетям почтовой связи.>> Исходя из данного закона, в дальнейшем под <<\textbf{тайной связи}>>  будет подразумеваться совокупность тайны переписки, телефонных разговоров, почтовых отправлений, телеграфных и иных сообщений, передаваемых по сетям электросвязи и сетям почтовой связи. Также, в дальнейшем к сетям электросвязи будем относить и Интернет.

Подобные законы существуют в большинстве развитых стран. Например, четвертая поправка  к Конституции США  гласит : <<Право народа на охрану личности, жилища, бумаг и имущества от необоснованных обысков и арестов не должно нарушаться. Ни один ордер не должен выдаваться иначе, как при наличии достаточного основания, подтвержденного присягой или торжественным заявлением; при этом ордер должен содержать подробное описание места, подлежащего обыску, лиц или предметов, подлежащих аресту>> %LITER: https://photos.state.gov/libraries/adana/30145/publications-other-lang/RUSSIAN.pdf

Нарушениями тайны связи не является :
\begin{itemize}
		\item Прослушивание (в том числе и обыск) без ордера   в случае проведения контрразведывательных операций. Однако подобное допустимо только при условии наличия достаточных оснований и обоснования того, почему в конкретном случае получение ордера не целесообразно. При этом правоохранительные органы могут искать лишь доказательства, подтверждающие факты действия разведывательных органов иностранных государств.
\end{itemize}

 Применимо к особенностям организации общения и передачи данных в Интернете, под   \textbf{ <<нарушением тайны связи}>> будут подразумеваться следующие ситуации : 
\begin{itemize}
	\item Передача стороной, предоставляющей услуги связи, данных об абонентах связи, времени связи и прочих параметрах сообщений третьим лицам
	\item  Проведение атаки на любые  физические компоненты коммуникационных сетей  : ЭВМ пользователей или стороны, предоставляющей услуги связи, сетевое оборудование, серверы; атаки типа <<Man in the middle>>, выполняемые непосредственно на линиях связи
	\item Использование правоохранительными органами прослушивающего оборудования без соответствующих санкций (ордера) суда или другие действия, выходящие за рамки полномочий правоохранительных органов, ведомств и силовых структур данного государства 
\end{itemize} 
 
\newpage % следующая глава с новой строки