\addcontentsline{toc}{section}{Список использованной литературы}
\begin{thebibliography}{}
	\bibitem{AntiMail1}
	Анализ угроз информационной безопасности 2016-2017, Anti-Malware [Электронный ресурс] / Режим доступа:   https://goo.gl/FeQbSC
	\bibitem{AntiMail2}
	Основные каналы утечки информации на предприятии, Anti-Malware [Электронный ресурс] / Режим доступа:   https://goo.gl/gJoQyE
	\bibitem{63FZ}
	Федеральный закон от 7 июля 2003 г. N 126-ФЗ
	"О связи", Система ГАРАНТ [Электронный ресурс] / Режим доступа:   https://goo.gl/vf2aww
	\bibitem{149FZ}
	Федеральный закон
	об информации, информационных  технологиях
и о защите информации от 27.07.2006
, КонсультантПлюс [Электронный ресурс] / Режим доступа: https://goo.gl/8btF7G
	\bibitem{4Popravka}
	Билль о правах
, U.S Department of State  [Электронный ресурс] / Режим доступа: https://goo.gl/8HFr5R
	\bibitem{EUDataProtec}	Защита персональных данных	в Евросоюзе и США, TAdviser  [Электронный ресурс] / Режим доступа: https://goo.gl/oG5Nmv
	\bibitem{MailWare1}	Как удалить SpyWare и что это такое, Заметки сис.админа  [Электронный ресурс] / Режим доступа: https://goo.gl/pnF2w7
	\bibitem{MailWare2}	Что такое Keylogger (кейлоггер) и как перехватывают данные ввода, Заметки сис.админа  [Электронный ресурс] / Режим доступа: https://goo.gl/P5XuEF
	\bibitem{INT16H}	Функции BIOS - INT 16H: сервис клавиатуры, CodeNet [Электронный ресурс] / Режим доступа: https://goo.gl/aySyG6
	\bibitem{Csrss}Csrss.exe: что это?, FB.ru [Электронный ресурс] / Режим доступа: https://goo.gl/Q8MKSc
	\bibitem{PressPerDay}Сколько в день жмет девелопер?, Хабр [Электронный ресурс] / Режим  доступа: https://goo.gl/D4Fxmo
	\bibitem{CryproAttackDiscribe}Атаки на криптографические протоколы -- доклад, MyUnivercity [Электронный ресурс] / Режим   доступа: https://goo.gl/EdMjJd
	\bibitem{SocialCredit}Planning Outline for the Construction of a Social Credit System (2014-2020), China Copyright \& Media [Электронный ресурс] / Режим доступа: https://goo.gl/BkvXZu
	\bibitem{SocialCredit2}Цифровая диктатура: как в Китае вводят систему социального рейтинга, РБК [Электронный ресурс] / Режим доступа: https://goo.gl/5Me5mE
	\bibitem{MAINWAY1}U.S. surveillance architecture includes collection of revealing Internet, phone metadata, The Wasghington Post [Электронный ресурс] / Режим доступа: https://goo.gl/CpUEbv
	\bibitem{MAINWAY2} Blowing in the Stellar Wind, The Starshollow Gazette [Электронный ресурс] / Режим доступа: https://goo.gl/FhTdKW
	\bibitem{614-1} AT\&T Whistle-Blower's Evidence, WIRED [Электронный ресурс] / Режим доступа: https://goo.gl/X3uvgJ
	\bibitem{614-2}  The NSA Is Building the Country’s Biggest Spy Center (Watch What You Say), WIRED [Электронный ресурс] / Режим доступа: https://goo.gl/mJXjnj
	\bibitem{SPIEGEL} The NSA Uses Powerful Toolbox in Effort to Spy on Global Networks, Der Spiegel [Электронный ресурс] / Режим доступа: https://goo.gl/R1hgBN
	\bibitem{TAO2} АНБ устанавливает бэкдоры на ноутбуки из интернет-магазинов, Хакер [Электронный ресурс] / Режим доступа: https://goo.gl/PLxgZU
	\bibitem{TAO3} How the U.S. Government Hacks the World, BloomBerg [Электронный ресурс] / Режим доступа: https://goo.gl/RL1uQn
	\bibitem{BoIn1} Meet 'Boundless Informant,' the NSA's Secret Tool for Tracking Global Surveillance Data, The Atlantic [Электронный ресурс] / Режим доступа: https://goo.gl/kgYy9z
	\bibitem{PRISM1}NSA Prism program taps in to user data of Apple, Google and others , The Guardian [Электронный ресурс] / Режим доступа: https://goo.gl/g2hYNM
	\bibitem{PRISM2} U.S., company officials: Internet surveillance does not indiscriminately mine data, The Wasghington Post [Электронный ресурс] / Режим доступа: https://goo.gl/yVHU9J
	\bibitem{PRISM3} Bitmessage's NSA-Proof E-Mail, Bloomberg [Электронный ресурс] / Режим доступа: https://goo.gl/YCiytf
	\bibitem{Narus} Company Overview of Narus Networks Private Limited, Bloomberg [Электронный ресурс] / Режим доступа: https://goo.gl/LYfPJy
	\bibitem{Tempora} The UK Tempora Program Captures Vast Amounts of Data — and Shares with NSA, The Atlantic [Электронный ресурс] / Режим доступа: https://goo.gl/zsfwNt
	\bibitem{MUSCULAR} How we know the NSA had access to internal Google and Yahoo cloud data, The Wasghington Post [Электронный ресурс] / Режим доступа: https://goo.gl/tdXEGb
	\bibitem{Onyx} Reporters cleared of revealing military secret, SwissInfo [Электронный ресурс] / Режим доступа: https://goo.gl/MNexxU
	\bibitem{SORM1} Постановление Правительства Российской Федерации от 27 августа 2005 г. N 538 г. Москва Об утверждении Правил взаимодействия операторов связи с уполномоченными государственными органами, осуществляющими оперативно-разыскную деятельность , Российская Газета  [Электронный ресурс] / Режим доступа: https://goo.gl/kB4q4J
	\bibitem{SORM2} Слушать подано ,  Газета  Коммерсантъ[Электронный ресурс] / Режим доступа: https://goo.gl/zAdbS6
	\bibitem{Yar1} Закон Яровой может быть на руку госкорпорации «Ростех»,    ВЕДОМОСТИ [Электронный ресурс] / Режим доступа: https://goo.gl/mciZmn
	\bibitem{Yar2} «Принцип асимметричной криптографии»: почему в Telegram отрицают возможность предоставить ФСБ ключи шифрования,    РТ на русском [Электронный ресурс] / Режим доступа: https://goo.gl/QqhKon
	\bibitem{TOR1} Towards	an	Analysis	of	Onion	Routing	Security,   cryptome [Электронный ресурс] / Режим доступа: https://goo.gl/hWnXcy
	\bibitem{TOR2}  PRISM против Tor,   PgpRu [Электронный ресурс] / Режим доступа: https://goo.gl/u3X8Lo
	\bibitem{I2P1}  Денис Колисниченко. Анонимность и безопасность в Интернете: от "чайника" к пользователю. — БХВ-Петербург, 2011. — С. 44, 46, 47. — 240 с. — ISBN 978-5-9775-0363-1.
	\bibitem{I2P2}Adrian Crenshaw. Darknets and hidden servers: Identifying the true IP/network identity of I2P service hosts // In the Proceedings of Black Hat 2011. — Washington, DC, 2011.
	\bibitem{Steg} Confounding Carnivore: How to Protect Your Online Privacy,   AlterNet [Электронный ресурс] / Режим доступа: https://goo.gl/GzfKZi
	\bibitem{PopApp} The Most Popular Messaging App in Every Country,   SimilarWeb [Электронный ресурс] / Режим доступа: https://www.similarweb.com/blog/worldwide-messaging-apps
	\bibitem{PopAppRus} Эксперты назвали самый популярный мессенджер в России,   РБК [Электронный ресурс] / Режим доступа: https://goo.gl/59wU6N
	\bibitem{Viber} Безопасность до блокировки доведет,   Коммерсантъ [Электронный ресурс] / Режим доступа: https://goo.gl/fGN1gf
	\bibitem{WuzUp} WhatsApp: Popular Free Messaging Service Puts Users At Risk,   WorldChurch [Электронный ресурс] / Режим доступа: https://goo.gl/kK1cby
	\bibitem{TG1} Telegram: новый мессенджер от Павла Дурова,   Slon [Электронный ресурс] / Режим доступа: https://goo.gl/9XUuVz
	\bibitem{TG2} A   practical   cryptanalysis   of   the	Telegram   messaging   protocol,   nourbakhsh [Электронный ресурс] / Режим доступа: https://goo.gl/KBkWek
	\bibitem{BF1} Schneier B. Applied Cryptography: Protocols, Algorithms, and Source Code in C — 2 — John Wiley \& Sons, 1996. — 784 p. — ISBN 978-0-471-12845-8, 978-0-471-11709-4
	\bibitem{BF2} Vaudenay S. On the Weak Keys of Blowfish // Fast Software Encryption: Third International Workshop Cambridge, UK, February 21–23 1996 Proceedings / D. Gollmann — Berlin: Springer Berlin Heidelberg, 1996. — P. 27–32. — (Lecture Notes in Computer Science; Vol. 1039) — ISBN 978-3-540-60865-3 — ISSN 0302-9743 — doi:10.1007/3-540-60865-6\_39
\end{thebibliography}