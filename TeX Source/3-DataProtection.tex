\parindent=1cm %красная строка
\begin{center}
		
		\section{Защита личной переписки}
		
\end{center}

Принимая во внимание большое число угроз, рассмотрим существующие правовые и фактические способы обеспечения секретности тайны связи.
\subsection{Способы защиты и ответственность в правовом аспекте}

Уже упомянутый Федеральный закон <<Об информации, информационных технологиях и о защите информации>>   вводит дисциплинарную, гражданско-правовую, административную или уголовную ответственность
 за   нарушение интересов и прав лиц, пострадавших от разглашения информации ограниченного доступа или любого другого неправомерного использования данной информации.   
%LITER: http://www.consultant.ru/cons/cgi/online.cgi?req=doc&base=LAW&n=221952&fld=134&dst=100144,0&rnd=0.40247948281489154#07281395619440368
%определяет набор правовых, организационных и технических мер,  целью которых является  защита информации от неправомерного доступа, модификации, блокирования, копирования и распространения. Также вводится ответственность за правонарушения в сфере информационных технологий и защиты информации.
\subsection{Защита переписки при помощи существующего ПО}

Основными причинами  утечки и перехвата сообщений являются SpyWare (вирусы), целенаправленные атаки на протоколы и средства связи, халатность отправителя, состоящая в использовании недоверенных  сетей и средств. Рассмотрим каждую причину подробнее.

\textbf{SpyWare}. Вирус в классическом понимании представляет собой программы, целенаправленно создающие свои копии и передающие их по разным каналам связи на другие устройства , способные внедряться в код других программ, загрузочные секторы жёстких дисков. При этом основной функцией вируса является саморепликация и распространение, а модификация  работы аппаратно-программных комплексов -- всего лишь сопутствующая функция. SpyWare (сокр от Spy Software -- <<Шпионское программное обеспечение>>) представляет отдельный класс вредоносного ПО, лишенный репликативных свойств вируса. Основным назначением SpyWare является мониторинг, сохранение и передача злоумышленнику данных о работе ПО, пользовательской активности и самом пользователе  на заражённом устройстве. Установка таких программ происходит скрытно и не предполагает возможности пользователя следить за работой такой программы или её удаления. К SpyWare не относятся программы, добровольно установленные пользователем и применяющиеся на совершенно  законных основаниях  для мониторинга состояния устройства, оказания удалённой технической поддержки,исследования защищённости компьютерных систем, желаемых пользователем персонализации и обновления компонентов ПО,     Для перехвата сообщений используются кейлоггеры(keyloggers), осуществляющие логирование всех нажатых клавиш, скрин-скраперы(screen scrapers), создающие снимки экрана через заданный интервал времени или по наступлению события, и обобщенные следящие программы, способные перехватывать содержимое почтовых программ и веб-страниц, открытых на заражённом устройстве, с помощью post-get запросов и автоматизированных средств взаимодействия с веб-браузером таких как Selenium.  %LITER https://ru.wikipedia.org/wiki/Spyware https://sonikelf.ru/chto-takoe-spyware-i-kak-s-etim-borotsya/
\newpage 