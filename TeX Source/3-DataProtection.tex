\parindent=1cm %красная строка
\begin{center}
		
		\section{Защита личной переписки}
		
\end{center}

Принимая во внимание большое число угроз, рассмотрим существующие правовые и фактические способы обеспечения секретности тайны связи.
\subsection{Способы защиты и ответственность в правовом аспекте}

Уже упомянутый Федеральный закон <<Об информации, информационных технологиях и о защите информации>>   вводит дисциплинарную, гражданско-правовую, административную или уголовную ответственность за   нарушение интересов и прав лиц, пострадавших от разглашения информации ограниченного доступа или любого другого неправомерного использования данной информации.   Подробности защиты тайны связи в России и мире рассмотрены в пункте 1.1 <<Понятия в правовом аспекте>>
%LITER: http://www.consultant.ru/cons/cgi/online.cgi?req=doc&base=LAW&n=221952&fld=134&dst=100144,0&rnd=0.40247948281489154#07281395619440368
%определяет набор правовых, организационных и технических мер,  целью которых является  защита информации от неправомерного доступа, модификации, блокирования, копирования и распространения. Также вводится ответственность за правонарушения в сфере информационных технологий и защиты информации.
\subsection{Защита переписки при помощи существующего ПО и его анализ}
Далее  рассмотрены существующие способы защиты тайны переписки в интернете с помощью существующего ПО, проведен детальный анализ и выбраны оптимальные средства для конкретных задач, т.е. оптимального баланса простоты использования, доступности и надёжности. Также рассмотрены методы защиты от угроз описанных  в пункте 2.2. 
\\


\textbf{Использование доверенного безопасного  ПО. } В первую очередь, для защиты частной переписки необходимо убедиться в использовании оригинальных программных продуктов, поставляемых надёжными поставщиками. В качестве критериев надёжности можно выбрать:
\begin{itemize}
	\item Популярность. Если продукт находится на рынке достаточно долго, имеет хорошие отзывы и нет известных инцидентов компрометации данного продукта, то такой продукт можно считать <<надёжным>>. 
	\item Получение из оригинальных источников. Используемый продукт необходимо получать только от доверенного поставщика, т.е. ПО должно быть получено от официального дистрибьютора и/или из доверенного источника (официальный сайт, репозиторий).	
	\item Проверка на оригинальность. Для защиты от реверс-инжиниринга и/или внедрения модификации в исполнимый файл или исходный код, если продукт распространяется в таком виде, необходимо использовать валидацию полученного     продукта с помощью хэш-функций, например MDA-5, SHA-256. Такой подход используется как для проприетарного    (пакет Office от Microsoft), так и для open-source ПО (wine, transmission, vim). В противном случае возможно изменение ПО для превращения в кейлоггер или аналогичную программу слежения. \\
\end{itemize}


\textbf{Средства анонимного или шифрованного общения: мессенджеры, ремейлеры, сетевые средства}. Анонимные оверлейные сети -- это сети, работающие поверх уже существующей и работающей сети. Рассмотрим такие примеры таких сетей:

\textbf{Tor}. Анониманя оверлейная сеть, использующая принцип <<луковой маршрутизации>> -- технология анонимного обмена информацией, использующая многократное шифрование и пересылку шифрованных данных через цепочку частных узлов. Идеи, связанные с ЛМ, впервые появились в конце 90-х годов XX века и активно применялись ВМС США. Основной принцип работы ЛМ и Tor как частного случая: маршрутизатор при старте сессии  передачи выбирает случайное число промежуточных маршрутизаторов, генерирует сообщение для каждого, шифруя их симметричным ключом и указывая для каждого маршрутизатора, какой маршрутизатор будет следующим на пути (структура, аналогичная односвязному списку); для получения симметричного ключа устанавливается начальное соединение с каждым промежуточным маршрутизатором и используется его открытый ключ; таким образом, передаваемые по сети сообщения имеют <<луковую>> структуру, где для получения доступа к содержимому сообщения необходимо поочередно <<снимать слои>> ; каждый маршрутизатор <<снимает один слой>>, получает предназначенные только ему указания маршрутизации (следующий прокси) и шифрованное сообщение, которое необходимо передать далее; последний маршрутизатор <<снимает последний слой>>, отправляет сообщение адресату. Таким образом формируется устойчивая сеть, где каждый прокси передает сообщения в любую сторону, наращивая слои шифрования при передаче ответного сообщения. %LITER http://www.inf.uni-konstanz.de/dbis/teaching/ss03/internet-protocols/download/onion.pdf
%LITER http://cryptome.org/2014/08/onion-routing-security-2000.pdf
%GRAPH https://upload.wikimedia.org/wikipedia/commons/d/dc/Tor-onion-network.png

Преимущества Tor и луковой маршрутизации:высокая степень несвязности сети, прямо зависящая от кол-ва участвующих прокси; возможность работы даже при наличии скомпрометированных узлов, если только вся сеть не стоит из таких узлов; сочетание Tor и других средств шифрования и анонимности позволяет бороться с PRISM. %LITER https://www.pgpru.com/novosti/2013/prismprotivtor


Недостатки: отсутствие защиты от анализа синхронизации в слабонагруженных сетях, отсутствие защиты   от анализа данных, проходящих через выходные узлы, т.к. оператор может получить доступ к данным через сниффинг, если только не используется конечная криптография типа SSL/TSL; уязвимости к атакам MITM,  по времени, по сторонним каналам, глобальному пассивному наблюдению;  %LITER https://www.pgpru.com/faq/anonimnostjobschievoprosy#h37444-7
%LITER https://webcourse.cs.technion.ac.il/236349/Spring2014/ho/WCFiles/2011-01-2.report.pdf
%LITER https://arstechnica.com/information-technology/2013/09/snoops-can-identify-tor-users-given-enough-time-experts-say/
%LITER https://www.freehaven.net/anonbib/topic.html
ошибки  в программной реализации; на последнем узле цепи Tor возможна деанонимизация отправителя или модификация отправляемого сообщения;  при работе с сетью  к сообщениям пользователя может добавляться техническая информация, полностью либо частично раскрывающая отправителя. %LITER https://xakep.ru/2014/10/27/tor-russia/

\textbf{Чесночная маршрутизация, I2P}.
\newpage 