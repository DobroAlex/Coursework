\parindent=1cm %красная строка
\begin{center}
	
	\section{Понятие <<цифровой угрозы>> , новые цифровые угрозы}
	
\end{center}

Дадим определение понятию << цифровая угроза >> и рассмотрим их основные виды.
\subsection{Определение}


	\textbf{Цифровая угроза} -- совокупность условий и факторов, создающих опасность нарушения информационной безопасности  в контексте нарушения тайны связи.%LITER: https://ru.wikipedia.org/wiki/%D0%A3%D0%B3%D1%80%D0%BE%D0%B7%D1%8B_%D0%B8%D0%BD%D1%84%D0%BE%D1%80%D0%BC%D0%B0%D1%86%D0%B8%D0%BE%D0%BD%D0%BD%D0%BE%D0%B9_%D0%B1%D0%B5%D0%B7%D0%BE%D0%BF%D0%B0%D1%81%D0%BD%D0%BE%D1%81%D1%82%D0%B8#%D0%9B%D0%B8%D1%82%D0%B5%D1%80%D0%B0%D1%82%D1%83%D1%80%D0%B0
<<Цифровая угроза>> является частным случаем \textit{угрозы информационной безопасности} -- угрозой конфиденциальности (неправомерный доступ к информации) и угрозой доступности (осуществление действий, делающих невозможным или затрудняющих доступ к ресурсам информационной системы). 


\subsection{Основные виды}

%Основными причинами  утечки и перехвата сообщений являются SpyWare (вирусы), целенаправленные атаки на протоколы и средства связи, халатность отправителя, состоящая в использовании недоверенных  сетей и средств. Рассмотрим каждую причину подробнее.
	Основными видами угроз являются  утечки и перехваты сообщений, происходящие с помощью  SpyWare (подмножество вирусов),  целенаправленных атак  на протоколы и средства связи,атак на криптографические протоколы, халатность отправителя, состоящая в использовании недоверенных  сетей и средств. Рассмотрим каждый вид  подробнее.
	
	\textbf{SpyWare}. Вирус в классическом понимании представляет собой программы, целенаправленно создающие свои копии и передающие их по разным каналам связи на другие устройства , способные внедряться в код других программ, загрузочные секторы жёстких дисков. При этом основной функцией вируса является саморепликация и распространение, а модификация  работы аппаратно-программных комплексов -- всего лишь сопутствующая функция. SpyWare (сокр от Spy Software -- <<Шпионское программное обеспечение>>) представляет отдельный класс вредоносного ПО, лишенный репликативных свойств вируса. Основным назначением SpyWare является мониторинг, сохранение и передача злоумышленнику данных о работе ПО, пользовательской активности и самом пользователе  на заражённом устройстве. Установка таких программ происходит скрытно и не предполагает возможности пользователя следить за работой такой программы или её удаления.     Для перехвата сообщений используются кейлоггеры(keyloggers), осуществляющие логирование всех нажатых клавиш, скрин-скраперы(screen scrapers), создающие снимки экрана через заданный интервал времени или по наступлению события, и обобщенные следящие программы, способные перехватывать содержимое почтовых программ и веб-страниц, открытых на заражённом устройстве, с помощью post-get запросов и автоматизированных средств взаимодействия с веб-браузером таких как Selenium. 
	
	
	К SpyWare не относятся программы, добровольно установленные пользователем и применяющиеся на совершенно  законных основаниях  для мониторинга состояния устройства, оказания удалённой технической поддержки,исследования защищённости компьютерных систем, желаемых пользователем персонализации и обновления компонентов ПО. %LITER https://ru.wikipedia.org/wiki/Spyware
	%LITER https://sonikelf.ru/chto-takoe-spyware-i-kak-s-etim-borotsya/
	
	Рассмотрим отдельно самого распространённого представителя SpyWare -- \textbf{кейлоггер} -- программный или аппаратный комплекс, регистрирующий взаимодействие пользователя с устройствами ввода-вывода, в классическом случае -- с клавиатурой и мышкой. %LITER https://sonikelf.ru/keylogger-chto-eto-ili-shpionazh-chistoj-vody-na-pk/ 
	Первые кейлоггеры появились в эпоху MS-DOS и представляли собой перехватчик прерывания int  16h.	 %LITER http://www.codenet.ru/progr/dos/int_0015.php
	
	
	Современные компьютеры, работающие в protected mode, не дают программисту доступ к таким низкоуровневым возможностям, поэтому теперь в основе современных кейлоггеров лежит использование \textbf{хуков} -- технологии, позволяющей изменить стандартное поведение тех или иных компонентов информационной системы. Обычно для этого используются компоненты Win32API: WH\_Keyboard, WH\_JOURNALRECORD. Преимущество последнего заключается в отсутствии необходимости использования DLL, что упрощает распространения вируса через компьютерные сети. Недостатком использования хуков является легкая обнаружимость DLL с хуком, так как для перехвата нажатий DLL отображается в адресное пространство всех GUI-процессов. 
	
	
	Второй популярной методикой является циклический опрос состояния клавиатуры с высокой скоростью. Преимуществом является меньшая заметность кейлоггера, однако присутствует значительный недостаток -- необходимость очень частого опроса клавиатуры, примерно 10-20 опросов в секунду -- современные ОС могут не выделить процессу с низким приоритетов столько ресурсов или не предоставлять доступ с такой частотой. 
	
	Третий способ является одним из наиболее эффективных и представляет собой кейлоггер уровня драйвера. В таком случае кейлоггер является частью драйвера, незаметен для большинства антивирусов, не может быть удален без потери функциональности клавиатуры. Также возможна реализация драйвера-фильтра, являющегося прослойкой между настоящим драйвером и ОС. Также к низкоуровневым кейлоггерам может быть отнесен руткит, перехватывающий обмен  csrss.exe (Server Client Runtime Subsystem)
	%LITER http://fb.ru/article/195605/csrss-exe---chto-eto-csrss-exe-gruzit-protsessor-kak-lechit
	%GRAPH:  https://sonikelf.ru/attach/img/1351669030-clip-9kb.jpg  
	
	В последнее время на рынке гаджетов появились аппаратные клавиатурные устройства, имеющие сходный с программным кейлоггером   функционал, представляющие собой USB-флешки, регистрирующие нажатия клавиш и записывающие их на собственную память. Такое устройство может автономно работать достаточно долго. Если предположить, что средний менеджер нажимает примерно 23000 клавиши в день(обозначим константой ApD), один символ занимает 1 килобайт памяти (обозначен переменной  S) %LITER https://habr.com/company/io/blog/263795/ 
	и взять емкость запоминающего устройства 16Gb (обозначим константой Mem), то памяти хватит на $ \frac{Mem}{Apd * S } = \frac{16 Gb}{23000*9,54*10^{-7} Gb} $ = 727 дней автономной работы. 
	
	\textbf{Атаки на протоколы и средства связи}
	
	
	Большинство атак на протоколы связи основаны на принципе \textbf{<<Человек в середине>>} или <<Атака посредника>>  (<<Man in the midle>>, MITM). В основе такой атаки лежит перехват сообщений на линии коммуникации между отправителем  и абонементом.  При этом возможны два метода атаки: пассивное прослушивание заключается в перехвате и анализе сообщений, если они зашифрованы, активная атака предполагает перехват, анализ сообщений, взлом криптографических алгоритмов, если такие используются, изменение содержимого сообщения и/или предотвращение передачи без разрушения канала связи. 
	
	Современные протоколы коммуникации используют различные криптографические протоколы, при этом шифрование происходит непосредственно на устройствах, то есть через коммуникационные сети передается уже зашифрованное сообщение, которое невозможно просто прочитать или модифицировать, не взломав ключ шифрования или не использовав другую уязвимость, поэтому будут рассмотрены именно активные методы атаки. 
	
	Пример атаки на алгоритмическом языке: Алиса хочет передать сообщение Бобу, Мэлори хочет перехватить и, возможно, изменить его так, чтобы Боб получил  злонамеренно ошибочное сообщение:
	\begin{enumerate}
		\item Алиса отправляет сообщение Бобу,  сообщение перехватывает Мэлори;
		\item Мэлори пересылает сообщение Бобу, который не знает, что сообщение не от Алисы;
		\item Боб посылает свой ключ;
		\item Мэлори подменяет ключ Боба своим, затем  пересылает сообщение Алисе;
		\item Алиса принимает сообщение, шифрует свое сообщение ключом Мэлори, который считает ключом Боба и  что только он сможет расшифровать его, отправляет сообщение Бобу;
		\item Мэлори перехватывает сообщение, шифрованное ключом Мэлори (лже-Боба), модифицирует его, шифрует ключом Боба и отправляет Бобу;
		\item Теперь Мэлори может модифицировать сообщения  обеих сторон, даже если те решат изменить ключи.
	\end{enumerate}
	Атаки типа MIT показывают важность точного подтверждения того, что обе стороны используют настоящие открытые ключи: у стороны A открытый ключ стороны B и у стороны B открытый ключ A. Если такое подтверждение не используется, то канал может быть атакован по принципу MIT. 
	
	\textbf{Атаки на криптографические протоколы}
	Криптографические протоколы в зависимости от сложности  решают одну или несколько   задач: шифрование/дешифрование, создание электронной цифровой подписи (ЭЦП, digital signature, DS), идентификация/аутентификация, аутентифицированного распределение ключей. Атаки на протоколы можно разделить на пассивные и активные: при пассивных атаках взломщик(криптоаналитик) не участвует в протоколах, только следит за протоколом и пытается раздобыть ценную информацию на основе перехватываемого шифротекста; при активных атаках   аналитик пытается изменить протокол к собственной выгоде и для этой цели активный взломщик может выдавать себя за другого человека, повторять или 	 заменять сообщения, разрывать линию, модифицировать информацию. В  целом, классификация атак на криптографические протоколы совпадает с классификацией атак на сетевые коммуникационные протоколы.
	
	Рассмотрим самые широко известные  атаки на криптографические протоколы:\\
	\textbf{Подмена}. Метод атаки заключается в подмене одного контрагента переписки другим. Аналитик,  выступая от имени одной стороны коммуникации, полностью имитирует её действия, получает сообщения определенного формата, необходимые для анализа шифротекста и подделки определенных шагов протокола.\\
	\textbf{Повторное навязывание сообщения} (replay attack). Атака основана на повторной передаче ранее переданных в текущей или прошедших сессиях  сообщений или частей сообщения. Например, повторная передача  информации проведенного ранее протокола идентификации/аутентификации может привести к повторной успешной идентификации/аутентификации атакующего как настоящего контрагента общения. Такая атака также может быть использована в протоколах передачи ключей для навязывания ранее использованного сеансового ключа и известна как атака на основе новизны (freshness attack).\\
	
\newpage