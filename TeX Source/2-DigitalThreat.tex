\parindent=1cm %красная строка
\begin{center}
	
	\section{Понятие <<цифровой угрозы>> , новые цифровые угрозы}
	
\end{center}

Дадим определение понятию << цифровая угроза >> и рассмотрим их основные виды.
\subsection{Определение}


	\textbf{Цифровая угроза} -- совокупность условий и факторов, создающих опасность нарушения информационной безопасности  в контексте нарушения тайны связи.%LITER: https://ru.wikipedia.org/wiki/%D0%A3%D0%B3%D1%80%D0%BE%D0%B7%D1%8B_%D0%B8%D0%BD%D1%84%D0%BE%D1%80%D0%BC%D0%B0%D1%86%D0%B8%D0%BE%D0%BD%D0%BD%D0%BE%D0%B9_%D0%B1%D0%B5%D0%B7%D0%BE%D0%BF%D0%B0%D1%81%D0%BD%D0%BE%D1%81%D1%82%D0%B8#%D0%9B%D0%B8%D1%82%D0%B5%D1%80%D0%B0%D1%82%D1%83%D1%80%D0%B0
<<Цифровая угроза>> является частным случаем \textit{угрозы информационной безопасности} -- угрозой конфиденциальности (неправомерный доступ к информации) и угрозой доступности (осуществление действий, делающих невозможным или затрудняющих доступ к ресурсам информационной системы). 


\subsection{Основные виды}

%Основными причинами  утечки и перехвата сообщений являются SpyWare (вирусы), целенаправленные атаки на протоколы и средства связи, халатность отправителя, состоящая в использовании недоверенных  сетей и средств. Рассмотрим каждую причину подробнее.
	Основными видами угроз являются  утечки и перехваты сообщений, происходящие с помощью  SpyWare (подмножество вирусов),  целенаправленных атак  на протоколы и средства связи, халатность отправителя, состоящая в использовании недоверенных  сетей и средств. Рассмотрим каждый вид  подробнее.
	
	\textbf{SpyWare}. Вирус в классическом понимании представляет собой программы, целенаправленно создающие свои копии и передающие их по разным каналам связи на другие устройства , способные внедряться в код других программ, загрузочные секторы жёстких дисков. При этом основной функцией вируса является саморепликация и распространение, а модификация  работы аппаратно-программных комплексов -- всего лишь сопутствующая функция. SpyWare (сокр от Spy Software -- <<Шпионское программное обеспечение>>) представляет отдельный класс вредоносного ПО, лишенный репликативных свойств вируса. Основным назначением SpyWare является мониторинг, сохранение и передача злоумышленнику данных о работе ПО, пользовательской активности и самом пользователе  на заражённом устройстве. Установка таких программ происходит скрытно и не предполагает возможности пользователя следить за работой такой программы или её удаления.     Для перехвата сообщений используются кейлоггеры(keyloggers), осуществляющие логирование всех нажатых клавиш, скрин-скраперы(screen scrapers), создающие снимки экрана через заданный интервал времени или по наступлению события, и обобщенные следящие программы, способные перехватывать содержимое почтовых программ и веб-страниц, открытых на заражённом устройстве, с помощью post-get запросов и автоматизированных средств взаимодействия с веб-браузером таких как Selenium. 
	К SpyWare не относятся программы, добровольно установленные пользователем и применяющиеся на совершенно  законных основаниях  для мониторинга состояния устройства, оказания удалённой технической поддержки,исследования защищённости компьютерных систем, желаемых пользователем персонализации и обновления компонентов ПО. %LITER https://ru.wikipedia.org/wiki/Spyware
	%LITER https://sonikelf.ru/chto-takoe-spyware-i-kak-s-etim-borotsya/
	
	Рассмотрим отдельно самого распространённого представителя SpyWare -- \textbf{кейлоггер} -- программный или аппаратный комплекс, регистрирующий взаимодействие пользователя с устройствами ввода-вывода, в классическом случае -- с клавиатурой и мышкой. %LITER https://sonikelf.ru/keylogger-chto-eto-ili-shpionazh-chistoj-vody-na-pk/ 
	 
\newpage