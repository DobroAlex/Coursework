\newpage
\parindent=1cm %красная строка? 
\begin{center}
	\addcontentsline{toc}{section}{Введение} %Убираем номер , даём имя в оглавлении 
	\section*{Введение} %сам текст заголовка 
	\pagestyle{plain} % нумерация выкл.
	\setcounter{page}{3} % начать нумерацию с номера тр
\end{center}

Актуальность работы  связана с возросшим числом новых угроз в области защиты личных данных, участившимися атаками частных лиц, группировок и специальных ведомств иностранных государств против частных лиц с целью получения частной информации, её анализа и использования для шантажа атакуемых лиц или использования в иных противозаконных целях. 

Целью данной работы является (%EDIT: сделать это в виде списка с пунктами?)
 анализ новых цифровых угроз, возникших в последнее десятилетие в связи с бурным развитием информационных технологий, за которым не последовал соразмерный рост знаний пользователей цифровых систем, используемые кибер-преступниками методы анализа и атаки на частные данные, правовой аспект защиты личной переписки, способы борьбы с ними в рамках существующего программного обеспечения, (%EDIT: заменить на ПО и создать список используемых сокращений до введения?)
разработка и реализация собственных алгоритмов для сохранения тайны личной переписки. 	