
\documentclass[a4paper,12pt,russian]{extreport}	%A4 бумага, 12 кегль, русский язык 
\usepackage{extsizes}
\usepackage{cmap} % для кодировки шрифтов в pdf
\usepackage[T2A]{fontenc}
%\usepackage{pscyr}
%\usepackage{graphicx} % для вставки картинок
\usepackage{makecell}
\usepackage{textcomp}
\usepackage{multirow} % улучшенное форматирование таблиц
\usepackage{ulem} % подчеркивания
\renewcommand{\rmdefault}{ftm} % Times New Roman
\usepackage[utf8]{inputenc}%включаем свою кодировку: koi8-r или utf8 в UNIX, cp1251 в Windows
%\usepackage[]{babel}	%больше поддержки русского языка 
\usepackage[english,russian, russianb]{babel}%используем русский и английский языки с переносами
\usepackage{amssymb,amsfonts,amsmath,mathtext,cite,enumerate,float} %подключаем нужные пакеты расширений
\usepackage[dvips]{graphicx} %хотим вставлять в диплом рисунки?
%\graphicspath{{images/}}%путь к рисункам

\makeatletter
\renewcommand{\@biblabel}[1]{#1.} % Заменяем библиографию с квадратных скобок на точку:
\makeatother

\usepackage{geometry} % Меняем поля страницы
\geometry{left=2cm}% левое поле
\geometry{right=1.5cm}% правое поле
\geometry{top=1cm}% верхнее поле
\geometry{bottom=2cm}% нижнее поле

\renewcommand{\theenumi}{\arabic{enumi}}% Меняем везде перечисления на цифра.цифра
\renewcommand{\labelenumi}{\arabic{enumi}}% Меняем везде перечисления на цифра.цифра
\renewcommand{\theenumii}{.\arabic{enumii}}% Меняем везде перечисления на цифра.цифра
\renewcommand{\labelenumii}{\arabic{enumi}.\arabic{enumii}.}% Меняем везде перечисления на цифра.цифра
\renewcommand{\theenumiii}{.\arabic{enumiii}}% Меняем везде перечисления на цифра.цифра
\renewcommand{\labelenumiii}{\arabic{enumi}.\arabic{enumii}.\arabic{enumiii}.}% Меняем везде перечисления на цифра.цифра
\renewcommand{\rmdefault}{ftm}

\begin{document}
	    \begin{titlepage}
    \newpage
	\pagestyle{empty} % нумерация выкл.
    \begin{center}
    \normalsize МИНИСТЕРСТВО ОБРАЗОВАНИЯ И НАУКИ РОССИЙСКОЙ ФЕДЕРАЦИИ\\ 
    \small  {Федеральное государственное автономное образовательное учреждение высшего образования} 
    \large \textbf{<<Крымский  федеральный  университет имени В. И. Вернадского>>} \\  \vspace{2mm}
    Таврическая академия (структурное подразделение ) \\
    \vspace{2mm}
    Факультет математики и информатики \\
    \vspace{2mm}
    Кафедра прикладной математики 
    \end{center}
    \vspace{3em}

    \begin{center}
	\normalsize Консманов Алексей Витальевич \\
    \LARGE \textbf{Сохранение тайны связи в условиях новых цифровых угроз} \\
    \vspace{1em}
    \normalsize Курсовая работа 
    \end{center}

    \vspace{1em}
    
    \begin{center}
    	\begin{tabbing}	%http://www.intuit.ru/studies/courses/1137/137/lecture/3835%3Fpage%3D5
    		\hspace{3cm}Обучающегося \hspace{3cm} \textbf{3 курса}\\ %Быдлокод?
    		\hspace{3cm}Направления подготовки \hspace{7mm}  \textbf{01.03.04}\\
    		\hspace{3cm}Форма обучения \hspace{26mm} \textbf{очная}
    	\end{tabbing}
    
	\vspace {3em}
    \flushleft Научный руководитель \hspace{20mm}  старший преподаватель 
    
    \hspace{75mm}кафедры прикладной математики  
    
    
    \hspace{75mm}В. А. Лушников
	\end{center}
    \vspace{\fill}

    \begin{center}
    Симферополь 2018
    \end{center}

    \end{titlepage}
% это титульный лист
	\tableofcontents % это оглавление, которое генерируется автоматически
\end{document}