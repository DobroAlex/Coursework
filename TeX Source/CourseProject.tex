
\documentclass[a4paper,14pt,russian]{extreport}	%A4 бумага, 14 кегль, русский язык 
\usepackage{extsizes}
\usepackage[onehalfspacing]{setspace} % поулторный интервал %https://proft.me/2013/06/9/latex-ukazanie-mezhstrochnogo-intervala

\usepackage{cmap} % для кодировки шрифтов в pdf
\usepackage[T2A]{fontenc}
%\usepackage{pscyr}
%\usepackage{graphicx} % для вставки картинок
\usepackage{mathptmx} %поддержка textbf
\usepackage{makecell}
\usepackage{textcomp}
\usepackage{multirow} % улучшенное форматирование таблиц
\usepackage{ulem} % подчеркивания

%полужирный шрифт http://tostudents.ru/2009/12/08/smena-shriftov-v-latex-tekst-i-formuly/
\renewcommand{\rmdefault}{ftm} % Times New Roman
\usepackage[utf8]{inputenc}%включаем свою кодировку: koi8-r или utf8 в UNIX, cp1251 в Windows
%\usepackage[]{babel}	%больше поддержки русского языка 
\usepackage[english,russian, russianb]{babel}%используем русский и английский языки с переносами
\usepackage{amssymb,amsfonts,amsmath,mathtext,cite,enumerate,float} %подключаем нужные пакеты расширений
\usepackage[dvips]{graphicx} %хотим вставлять в диплом рисунки?
\usepackage{cmap} % Улучшенный поиск русских слов в полученном pdf-файле
%\graphicspath{{images/}}%путь к рисункам
\usepackage{fancyhdr}%оформление нумерации 
\usepackage{tableof} %поддержка табличек
\usepackage{mathptmx}%
\usepackage{anyfontsize}% http://texblog.org/2012/08/29/changing-the-font-size-in-latex/
\usepackage{t1enc}%
\usepackage{cite}
\makeatletter
\renewcommand{\@biblabel}[1]{#1.} % Заменяем библиографию с квадратных скобок на точку:
\makeatother

\usepackage{geometry} % Меняем поля страницы
%QUEST: 3cm или 2 ? Ес\usepackage{cmap} % Улучшенный поиск русских слов в полученном pdf-файлели 3, придется менять форматирование заголовка 
\geometry{left=3cm}% левое поле
\geometry{right=2cm}% правое поле
\geometry{top=2cm}% верхнее поле
\geometry{bottom=2cm}% нижнее поле


\renewcommand{\theenumi}{\arabic{enumi}}% Меняем везде перечисления на цифра.цифра
\renewcommand{\labelenumi}{\arabic{enumi}}% Меняем везде перечисления на цифра.цифра
\renewcommand{\theenumii}{.\arabic{enumii}}% Меняем везде перечисления на цифра.цифра
\renewcommand{\labelenumii}{\arabic{enumi}.\arabic{enumii}.}% Меняем везде перечисления на цифра.цифра
\renewcommand{\theenumiii}{.\arabic{enumiii}}% Меняем везде перечисления на цифра.цифра
\renewcommand{\labelenumiii}{\arabic{enumi}.\arabic{enumii}.\arabic{enumiii}.}% Меняем везде перечисления на цифра.цифра

\renewcommand{\rmdefault}{ftm}
%NB: три команды ниже переопределяют некотрые шрифты  и дают поддержку жирного и прочиах текстов https://www.linux.org.ru/forum/general/4219163
\renewcommand{\rmdefault}{cmr} % Шрифт с засечками
\renewcommand{\sfdefault}{cmss} % Шрифт без засечек
\renewcommand{\ttdefault}{cmtt} % Моноширинный шрифт
\renewcommand*\thesection{\arabic{section}}
    
\begin{document}
	%\pagestyle{empty} % нумерация выкл.
	    \begin{titlepage}
    \newpage
	\pagestyle{empty} % нумерация выкл.
    \begin{center}
    \normalsize МИНИСТЕРСТВО ОБРАЗОВАНИЯ И НАУКИ РОССИЙСКОЙ ФЕДЕРАЦИИ\\ 
    \small  {Федеральное государственное автономное образовательное учреждение высшего образования} 
    \large \textbf{<<Крымский  федеральный  университет имени В. И. Вернадского>>} \\  \vspace{2mm}
    Таврическая академия (структурное подразделение ) \\
    \vspace{2mm}
    Факультет математики и информатики \\
    \vspace{2mm}
    Кафедра прикладной математики 
    \end{center}
    \vspace{3em}

    \begin{center}
	\normalsize Консманов Алексей Витальевич \\
    \LARGE \textbf{Сохранение тайны связи в условиях новых цифровых угроз} \\
    \vspace{1em}
    \normalsize Курсовая работа 
    \end{center}

    \vspace{1em}
    
    \begin{center}
    	\begin{tabbing}	%http://www.intuit.ru/studies/courses/1137/137/lecture/3835%3Fpage%3D5
    		\hspace{3cm}Обучающегося \hspace{3cm} \textbf{3 курса}\\ %Быдлокод?
    		\hspace{3cm}Направления подготовки \hspace{7mm}  \textbf{01.03.04}\\
    		\hspace{3cm}Форма обучения \hspace{26mm} \textbf{очная}
    	\end{tabbing}
    
	\vspace {3em}
    \flushleft Научный руководитель \hspace{20mm}  старший преподаватель 
    
    \hspace{75mm}кафедры прикладной математики  
    
    
    \hspace{75mm}В. А. Лушников
	\end{center}
    \vspace{\fill}

    \begin{center}
    Симферополь 2018
    \end{center}

    \end{titlepage}
% это титульный лист

	\tableofcontents % это оглавление, которое генерируется автоматически
	\thispagestyle{empty}%отключает нумерование страниц до введения включительно 
	%\addcontentsline{toc}{section}{Введение}% будет костыльно выглядеть
	\newpage
\parindent=1cm %красная строка? 
\begin{center}
	\addcontentsline{toc}{section}{Введение} %Убираем номер , даём имя в оглавлении 
	\section*{Введение} %сам текст заголовка 
	\pagestyle{plain} % нумерация выкл.
	\setcounter{page}{3} % начать нумерацию с номера тр
\end{center}

Актуальность работы  связана с возросшим числом новых угроз в области защиты личных данных, участившимися атаками частных лиц, группировок и специальных ведомств иностранных государств против частных лиц с целью получения частной информации, анализа полученных личных данных   и использования для шантажа атакуемых лиц, продажи или другого выгодного обмена, а также  в иных противозаконных целях. 

Целью данной работы является %EDIT: сделать это в виде списка с пунктами?
 анализ новых цифровых угроз, возникших в последнее десятилетие в связи с бурным развитием информационных технологий, за которым не последовал соразмерный рост знаний пользователей цифровых систем, используемые кибер-преступниками методы анализа и атаки на частные данные, правовой аспект защиты личной переписки, способы борьбы с угрозами  в рамках существующего программного обеспечения, %EDIT: заменить на ПО и создать список используемых сокращений до введения?
 сравнительный анализ существующих продуктов, разработка и реализация собственных алгоритмов для сохранения тайны личной переписки. 	 %введение
	\parindent=1cm %красная строка? 
\begin{center}
		
		\section{Понятия <<тайна связи>> и <<личная переписка>> в правом и информационном аспектах}
		
\end{center}

\subsection{Понятия в правовом аспекте} 

Так как все пользователь информационных систем являются в первую очередь гражданами правовых государств и объектами и субъектами права, рассмотрение основных понятий начнём с правого аспекта вопроса.

Согласно статье 63 федерального закона <<О связи>>: %LITER: http://home.garant.ru/#/document/186117/paragraph/645:1
<<На территории Российской Федерации гарантируется тайна переписки, телефонных переговоров, почтовых отправлений, телеграфных и иных сообщений, передаваемых по сетям электросвязи и сетям почтовой связи.>> Исходя из данного закона, в дальнейшем под <<\textbf{тайной связи}>>  будет подразумеваться совокупность тайны переписки, телефонных разговоров, почтовых отправлений, телеграфных и иных сообщений, передаваемых по сетям электросвязи и сетям почтовой связи. Также, в дальнейшем к сетям электросвязи будем относить и Интернет. 


 
\newpage % следующая глава с новой строки
	\parindent=1cm %красная строка
\begin{center}
	
	\section{Понятие <<цифровой угрозы>> , новые цифровые угрозы}
	
\end{center}

Дадим определение понятию << цифровая угроза >> и рассмотрим их основные виды.
\subsection{Определение}


	\textbf{Цифровая угроза} -- совокупность условий и факторов, создающих опасность нарушения информационной безопасности  в контексте нарушения тайны связи.%LITER: https://ru.wikipedia.org/wiki/%D0%A3%D0%B3%D1%80%D0%BE%D0%B7%D1%8B_%D0%B8%D0%BD%D1%84%D0%BE%D1%80%D0%BC%D0%B0%D1%86%D0%B8%D0%BE%D0%BD%D0%BD%D0%BE%D0%B9_%D0%B1%D0%B5%D0%B7%D0%BE%D0%BF%D0%B0%D1%81%D0%BD%D0%BE%D1%81%D1%82%D0%B8#%D0%9B%D0%B8%D1%82%D0%B5%D1%80%D0%B0%D1%82%D1%83%D1%80%D0%B0
<<Цифровая угроза>> является частным случаем \textit{угрозы информационной безопасности} -- угрозой конфиденциальности (неправомерный доступ к информации) и угрозой доступности (осуществление действий, делающих невозможным или затрудняющих доступ к ресурсам информационной системы). 

Угрозы называются <<\textbf{новыми}>>, так как их бурное развитие и рост числа инцидентов произошли в последние 10-15 лет и сами угрозы постоянно меняются, увеличивается их количество.  

\subsection{Основные виды, угрозы в частном секторе}

%Основными причинами  утечки и перехвата сообщений являются SpyWare (вирусы), целенаправленные атаки на протоколы и средства связи, халатность отправителя, состоящая в использовании недоверенных  сетей и средств. Рассмотрим каждую причину подробнее.
	Основными видами новых цифровых  угроз являются  утечки и перехваты сообщений, происходящие с помощью  SpyWare (подмножество вирусов),  целенаправленных атак  на протоколы и средства связи,атак на криптографические протоколы, халатность отправителя, состоящая в использовании недоверенных  сетей и средств. Рассмотрим каждый вид  подробнее.
	
	\textbf{SpyWare}. Вирус в классическом понимании представляет собой программы, целенаправленно создающие свои копии и передающие их по разным каналам связи на другие устройства , способные внедряться в код других программ, загрузочные секторы жёстких дисков. При этом основной функцией вируса является саморепликация и распространение, а модификация  работы аппаратно-программных комплексов -- всего лишь сопутствующая функция. SpyWare (сокр от Spy Software -- <<Шпионское программное обеспечение>>) представляет отдельный класс вредоносного ПО, лишенный репликативных свойств вируса. Основным назначением SpyWare является мониторинг, сохранение и передача злоумышленнику данных о работе ПО, пользовательской активности и самом пользователе  на заражённом устройстве. Установка таких программ происходит скрытно и не предполагает возможности пользователя следить за работой такой программы или её удаления.     Для перехвата сообщений используются кейлоггеры(keyloggers), осуществляющие логирование всех нажатых клавиш, скрин-скраперы(screen scrapers), создающие снимки экрана через заданный интервал времени или по наступлению события, и обобщенные следящие программы, способные перехватывать содержимое почтовых программ и веб-страниц, открытых на заражённом устройстве, с помощью post-get запросов и автоматизированных средств взаимодействия с веб-браузером таких как Selenium. 
	
	
	К SpyWare не относятся программы, добровольно установленные пользователем и применяющиеся на совершенно  законных основаниях  для мониторинга состояния устройства, оказания удалённой технической поддержки,исследования защищённости компьютерных систем, желаемых пользователем персонализации и обновления компонентов ПО. %LITER https://ru.wikipedia.org/wiki/Spyware
	%LITER https://sonikelf.ru/chto-takoe-spyware-i-kak-s-etim-borotsya/
	
	Рассмотрим отдельно самого распространённого представителя SpyWare -- \textbf{кейлоггер} -- программный или аппаратный комплекс, регистрирующий взаимодействие пользователя с устройствами ввода-вывода, в классическом случае -- с клавиатурой и мышкой. %LITER https://sonikelf.ru/keylogger-chto-eto-ili-shpionazh-chistoj-vody-na-pk/ 
	Первые кейлоггеры появились в эпоху MS-DOS и представляли собой перехватчик прерывания int  16h.	 %LITER http://www.codenet.ru/progr/dos/int_0015.php
	
	
	Современные компьютеры, работающие в protected mode, не дают программисту доступ к таким низкоуровневым возможностям, поэтому теперь в основе современных кейлоггеров лежит использование \textbf{хуков} -- технологии, позволяющей изменить стандартное поведение тех или иных компонентов информационной системы. Обычно для этого используются компоненты Win32API: WH\_Keyboard, WH\_JOURNALRECORD. Преимущество последнего заключается в отсутствии необходимости использования DLL, что упрощает распространения вируса через компьютерные сети. Недостатком использования хуков является легкая обнаружимость DLL с хуком, так как для перехвата нажатий DLL отображается в адресное пространство всех GUI-процессов. 
	
	
	Второй популярной методикой является циклический опрос состояния клавиатуры с высокой скоростью. Преимуществом является меньшая заметность кейлоггера, однако присутствует значительный недостаток -- необходимость очень частого опроса клавиатуры, примерно 10-20 опросов в секунду -- современные ОС могут не выделить процессу с низким приоритетов столько ресурсов или не предоставлять доступ с такой частотой. 
	
	Третий способ является одним из наиболее эффективных и представляет собой кейлоггер уровня драйвера. В таком случае кейлоггер является частью драйвера, незаметен для большинства антивирусов, не может быть удален без потери функциональности клавиатуры. Также возможна реализация драйвера-фильтра, являющегося прослойкой между настоящим драйвером и ОС. Также к низкоуровневым кейлоггерам может быть отнесен руткит, перехватывающий обмен  csrss.exe (Server Client Runtime Subsystem)
	%LITER http://fb.ru/article/195605/csrss-exe---chto-eto-csrss-exe-gruzit-protsessor-kak-lechit
	%GRAPH:  https://sonikelf.ru/attach/img/1351669030-clip-9kb.jpg  
	
	В последнее время на рынке гаджетов появились аппаратные клавиатурные устройства, имеющие сходный с программным кейлоггером   функционал, представляющие собой USB-флешки, регистрирующие нажатия клавиш и записывающие их на собственную память. Такое устройство может автономно работать достаточно долго. Если предположить, что средний менеджер нажимает примерно 23000 клавиши в день(обозначим константой ApD), один символ занимает 1 килобайт памяти (обозначен переменной  S) %LITER https://habr.com/company/io/blog/263795/ 
	и взять емкость запоминающего устройства 16Gb (обозначим константой Mem), то памяти хватит на $ \frac{Mem}{Apd * S } = \frac{16 Gb}{23000*9,54*10^{-7} Gb} $ = 727 дней автономной работы. 
	
	\textbf{Атаки на протоколы и средства связи}
	
	
	Большинство атак на протоколы связи основаны на принципе \textbf{<<Человек в середине>>} или <<Атака посредника>>  (<<Man in the midle>>, MITM). В основе такой атаки лежит перехват сообщений на линии коммуникации между отправителем  и абонементом.  При этом возможны два метода атаки: пассивное прослушивание заключается в перехвате и анализе сообщений, если они зашифрованы, активная атака предполагает перехват, анализ сообщений, взлом криптографических алгоритмов, если такие используются, изменение содержимого сообщения и/или предотвращение передачи без разрушения канала связи. 
	
	Современные протоколы коммуникации используют различные криптографические протоколы, при этом шифрование происходит непосредственно на устройствах, то есть через коммуникационные сети передается уже зашифрованное сообщение, которое невозможно просто прочитать или модифицировать, не взломав ключ шифрования или не использовав другую уязвимость, поэтому будут рассмотрены именно активные методы атаки. 
	
	Пример атаки на алгоритмическом языке: Алиса хочет передать сообщение Бобу, Мэлори хочет перехватить и, возможно, изменить его так, чтобы Боб получил  злонамеренно ошибочное сообщение:
	\begin{enumerate}
		\item Алиса отправляет сообщение Бобу,  сообщение перехватывает Мэлори;
		\item Мэлори пересылает сообщение Бобу, который не знает, что сообщение не от Алисы;
		\item Боб посылает свой ключ;
		\item Мэлори подменяет ключ Боба своим, затем  пересылает сообщение Алисе;
		\item Алиса принимает сообщение, шифрует свое сообщение ключом Мэлори, который считает ключом Боба и  что только он сможет расшифровать его, отправляет сообщение Бобу;
		\item Мэлори перехватывает сообщение, шифрованное ключом Мэлори (лже-Боба), модифицирует его, шифрует ключом Боба и отправляет Бобу;
		\item Теперь Мэлори может модифицировать сообщения  обеих сторон, даже если те решат изменить ключи.
	\end{enumerate}
	Атаки типа MIT показывают важность точного подтверждения того, что обе стороны используют настоящие открытые ключи: у стороны A открытый ключ стороны B и у стороны B открытый ключ A. Если такое подтверждение не используется, то канал может быть атакован по принципу MIT. 
	
	\textbf{Атаки на криптографические протоколы}
	Криптографические протоколы в зависимости от сложности  решают одну или несколько   задач: шифрование/дешифрование, создание электронной цифровой подписи (ЭЦП, digital signature, DS), идентификация/аутентификация, аутентифицированного распределение ключей. Атаки на протоколы можно разделить на пассивные и активные: при пассивных атаках взломщик(криптоаналитик) не участвует в протоколах, только следит за протоколом и пытается раздобыть ценную информацию на основе перехватываемого шифротекста; при активных атаках   аналитик пытается изменить протокол к собственной выгоде и для этой цели активный взломщик может выдавать себя за другого человека, повторять или 	 заменять сообщения, разрывать линию, модифицировать информацию. В  целом, классификация атак на криптографические протоколы совпадает с классификацией атак на сетевые коммуникационные протоколы.
	
	Рассмотрим самые широко известные  атаки на криптографические протоколы: %LITER:http://myunivercity.ru/%D0%9F%D1%80%D0%BE%D0%B3%D1%80%D0%B0%D0%BC%D0%BC%D0%B8%D1%80%D0%BE%D0%B2%D0%B0%D0%BD%D0%B8%D0%B5_%D0%B8_%D0%BA%D0%BE%D0%BC%D0%BF%D1%8C%D1%8E%D1%82%D0%B5%D1%80%D1%8B/%D0%90%D1%82%D0%B0%D0%BA%D0%B8_%D0%BD%D0%B0_%D0%BA%D1%80%D0%B8%D0%BF%D1%82%D0%BE%D0%B3%D1%80%D0%B0%D1%84%D0%B8%D1%87%D0%B5%D1%81%D0%BA%D0%B8%D0%B5_%D0%BF%D1%80%D0%BE%D1%82%D0%BE%D0%BA%D0%BE%D0%BB%D1%8B/361_43116_%D1%81%D1%82%D1%80%D0%B0%D0%BD%D0%B8%D1%86%D0%B01.html
	
	
	\textbf{Подмена}. Метод атаки заключается в подмене одного контрагента переписки другим. Аналитик,  выступая от имени одной стороны коммуникации, полностью имитирует её действия, получает сообщения определенного формата, необходимые для анализа шифротекста и подделки определенных шагов протокола.
	
	\textbf{Повторное навязывание сообщения} (replay attack). Атака основана на повторной передаче ранее переданных в текущей или прошедших сессиях  сообщений или частей сообщения. Например, повторная передача  информации проведенного ранее протокола идентификации/аутентификации может привести к повторной успешной идентификации/аутентификации атакующего как настоящего контрагента общения. Такая атака также может быть использована в протоколах передачи ключей для навязывания ранее использованного сеансового ключа и известна как атака на основе новизны (freshness attack).
	
	\textbf{Параллельная атака} (parallel-session attack). Аналитик открывает несколько параллельных сессий, при этом сообщения и полученные аналитиком  данные  из одного сеанса используются для   анализа шифротекста и ключей другого сеанса.
	
	\textbf{Атака с использованием специально подобранных текстов}. Атака на post-get запросы, при которой аналитик по определенному правилу подбирает запросы и их содержимое с целью анализа долговременного ключа собеседника.
	
	\textbf{Атака по известному сеансовому ключу} (known-key attack). Заключается в получении долговременных ключей, новых сессионных ключей или установлении алгоритма, используемого для генерации новых ключей по известному использованному ранее сессионному ключу.
	
	\textbf{Использование уязвимостей  алгоритма или ошибок реализации }. 	В атаках такого типа аналитик ищет уязвимости, связанные с алгоритмом или ошибки реализации. Такая атака может давать самые долговременные и серьезные результаты, так как для успешного отражения контрагентам необходимо узнать о компрометации используемого алгоритма и внести исправления в алгоритм и его реализацию. Однако, такая атака является достаточно затруднительной для аналитика, так как требует реверс-инжиниринга  алгоритма и его реализации, что может быть затруднительно в реальных коммуникационных сетях, где аналитику доступен только шифротекст и время отправки сообщения.\\
	
	Перечисленные выше угрозы относятся к частному и корпоративному общению, при этом государство или несколько государств не являются  стороной коммуникации или криптоаналитиком. Ниже рассмотрим теоретические ситуации и конкретные прецеденты, когда государство (под <<государством>> далее понимается совокупность судебной, законодательной и исполнительной властей конкретного государства) является криптоаналитиком или пособником аналитика по отношению к своим или иностранным гражданам. 
	\subsection{Государство как атакующая сторона}
	
	Современные законотворческие инициативы  ведущих стран Европы, Америки и Азии содержат в себе идею борьбы с терроризмом, опасность которого действительно невозможно не заметить, посредством массовой перлюстрации (просмотр личной пересылаемой корреспонденции, совершаемый втайне от контрагентов) или явного анализа цифровой переписки. 
	
	Основной проблемой в контексте тайны связи является возможность недобросовестного использования полученных данных с целью шантажа или  продажи, утечки из государственных информационных систем и хранилищ. Вторым большим опасением можно считать тот факт, что для доступа к переписке разработчики ПО оставляют backdoor'ы -- дефект алгоритма,  намеренно встраиваемый  в него разработчиком и позволяющий  получить несанкционированный доступ к данным, и если такой backdoor существует, то нет никаких гарантий, что доступ к нему не будет получен третьими лицами, что попадает под <<нарушение тайны связи>> из части 1 данной работы.

	Рассмотрим такие ситуации на примерах крупнейших мировых государств:
		
	 \textbf{Китай}. КНР проводит политику массовой слежки за гражданами по всей территории страны. В рамках этой политики реализованы два масштабных проекта: 
	
	\textbf{<<Золотой щит>>} (Великий китайский фаервол). Представляет собой  систему фильтрации содержимого интернета. Разрабатывался с 1998, введен в эксплуатацию повсеместно с 2003. Включает подсистемы управления безопасностью, информирования о правонарушениях, контроля входа и выхода, мониторинга и управления трафиком.  Функции проекта: ограничения доступа к ряду иностранных сайтов, ограничение публикаций для китайских СМИ, перехват и хранение сообщений в мессенджерах.
	
	\textbf{Интернет-цензура}. Включает в себя вышеописанный <<Золотой щит>> и комплекс мер, используемых правительством КНР для перехвата сообщений в мессенджерах и других средствах коммуникации, использующих защиту данных. 	Помимо использования стандартных приемов слежки, власти КНР используют социальные приемы. Так была создана социальная сеть <<Sina Weibo>> -- собственный проект китайской компанией Sina Corp в 2009 году. Существует мнение, что Sina Corp предоставляет доступ к данным и переписке пользователей правительству КНР, при этом сами пользователи считают, что их данные находятся под надёжной защитой владельцев платформы. 
	
	Параллельно с двумя вышеупомянутыми проектами в КНР в тестовом режиме работает система \textbf{<<Система социального кредита>>} -- система постоянного анализа поведения граждан в Интернете и в повседневной жизни. При этом в качестве поощрения и наказания выступают разрешение на работу в госучреждениях, возможность получать соцобеспечение, повышенное внимание таможни, возможность покупки билетов на самолеты и поезда, право на обучение детей в частных дорогих школах.   %LITER: https://chinacopyrightandmedia.wordpress.com/2014/06/14/planning-outline-for-the-construction-of-a-social-credit-system-2014-2020/ 
	%LITER: https://www.rbc.ru/business/11/12/2016/584953bb9a79477c8a7c08a7
	
	\textbf{США}. Несмотря на значительный уровень свободы слова и уважения к частной жизни, декларируемые Конституцией США, в государстве имеется широкая и мощная сеть компьютерного слежения и радиоэлектронной разведки. Задачи разведки  и слежения возложены на АНБ, ФБР, ЦРУ, Министерство финансов, Министерство обороны, Министерство юстиции и Министерство внутренней безопасности США.  
	
	Рассмотрим подробнее основные программы слежения США:
	
	\textbf{MAINWAY}. База данных, содержащая метаданные о нескольких миллиардах телефонных звонков, совершенных через самые крупные коммуникационные компании США: AT\&T, SBC, BellSouth, Verizon. Проект находится в ведении  АНБ и имеет несколько дочерних: Stellar Wing -- программа слежения за электронной активностью, в том числе электронной почтой, телефонными звонками, активностью в Интернете, имевшая место во времена президентства Дж. Буша-младшего(2001-2009 г-г);%LITER: https://www.washingtonpost.com/investigations/us-surveillance-architecture-includes-collection-of-revealing-internet-phone-metadata/2013/06/15/e9bf004a-d511-11e2-b05f-3ea3f0e7bb5a_story.html?utm_term=.1ef7b5fd28c8 										%LITER:	http://thestarshollowgazette.com/2012/08/25/blowing-in-the-stellar-wind/ 	  	%LITER: https://www.webcitation.org/6I1ZBmfih
	Комната 614А -- помещение в здании провайдера AT\&T, используемое АНБ в 2003-2006 годах для перехвата интернет-коммуникаций, %LITER: https://www.wired.com/2006/05/att-whistle-blowers-evidence/								%LITER:https://www.wired.com/threatlevel/2012/03/ff_nsadatacenter/all/1
	основной принцип работы перехватывающего оборудования -- разделение  оптического сигнала, при котором 90\% мощности  используются в дальнейшем в коммуникационном оборудовании и 10\% перенаправляются на порты мониторинга для изучения и записи.
	
	\textbf{Tailored Access Operations}. Подразделение АНБ, созданное в 1997 для пассивного и активного (взломы учетных записей, установка следящего оборудования, слежка за интернет-активностью) наблюдения за компьютерами. По данным Der Spiegel, перехватывающая спообность составляет 2 петабайта данных в час. %LITER http://www.spiegel.de/international/world/the-nsa-uses-powerful-toolbox-in-effort-to-spy-on-global-networks-a-940969.html?_gclid=5b0033e2a361e5.91658653-5b0033e2a36260.91306967&_utm_source=xakep&_utm_campaign=mention50611&_utm_medium=inline&_utm_content=lnk212297961700
	%LITER: https://xakep.ru/2013/12/30/61829/
	%LITER:https://www.bloomberg.com/news/articles/2013-05-23/how-the-u-dot-s-dot-government-hacks-the-world   
	Среди используемых методов слежки: перехват ноутбуков, отправленных почтой из интернет-магазинов, перехват сообщений о случаях сбоев ОС Windows. Согласно бюджетному плану, TAO следит за 85000 устройств по всему миру, имеет базы в США и Дармштадте, ФРГ. 
	
	\textbf{Boundless Informant}. Система анализа, обработки, хранения и визуализации массивов Big Data для анализа глобальных электронных коммуникаций.  Уже в начале 2013 года система хранила более 14 млрд. записей по Ирану, 6 млрд. --  по Индии и еще 2.8 млрд. записей по США. %LITER https://www.theatlantic.com/technology/archive/2013/06/meet-boundless-informant-the-nsas-secret-tool-for-tracking-global-surveillance-data/276686/
	В противоположность большинству проектов, требующих значительных финансовых вливаний в разработку ПО и инфраструктуры, BI использует готовые бесплатные open-source продукты, например, Google MapReduce и  Apache Hadoop Distributed File System.
	
	\textbf{PRISM}.  Государственная программа и 	комплекс мероприятий, осуществляемых с целью  негласного массового  сбора информации, передаваемой   сетями электросвязи, принятая АНБ в 2007. Утечка данных о существовании программы стало известно в 2013 после публикации отрывков секретной презентации в   <<The Guardian>> и <<The Washington Post>>. Мощность системы оценивается в 1.7 млрд. телефонных звонков и электронных сообщений и 5 млрд. записей о местонахождении владельцев мобильных телефонов в день.  %LITER https://www.theguardian.com/world/2013/jun/06/us-tech-giants-nsa-data
	%LITER https://www.washingtonpost.com/world/national-security/us-company-officials-internet-surveillance-does-not-indiscriminately-mine-data/2013/06/08/5b3bb234-d07d-11e2-9f1a-1a7cdee20287_story.html
	
	Обнародование информации о PRISM вызвало рост внимания общетсвенности на технологиях PGP, шифрованном мессенджере   Bitmessage и технологиях TOR. %LITER https://www.bloomberg.com/news/articles/2013-06-27/bitmessages-nsa-proof-e-mail
	
	
	\textbf{NarusInsight}. Кластерная система шпионажа, разрабатываемая компанией Boeing для американского правительства. %LITER https://www.bloomberg.com/research/stocks/private/snapshot.asp?privcapId=59181865
	Система состоит из большого количества компьютеров, соединённых в кластер и устанавливаемых в дата-центрах провайдеров интернета в США и Западной Европе. Система предоставляет очень широкие возможности для мониторинга, перехвата, хранения и анализа больших объёмов интернет-данных: масштабирование для анализа  сверхбольших IP-сетей, real-time обработку пакетов,  глубокая обработка данных искусственным интеллектом: нормализация, корреляция, агрегация и анализ, создающие информационные модели как отдельных пользователей, так и элементов информационных систем и их протоколов и приложений с возможностью анализа моделей в реальном времени; отслеживание индивидуальных пользователей и определение используемых  ими программ коммуникации, высокая надёжность и отказоустойчивость, может использоваться для блокировки шифрованных сетей,  построена в соответствии с законами о мониторинге пользователей CALEA и ETSI.
\newpage
	\parindent=1cm %красная строка
\begin{center}
		
		\section{Защита личной переписки}
		
\end{center}

Принимая во внимание большое число угроз, рассмотрим существующие правовые и фактические способы обеспечения секретности тайны связи.
\subsection{Способы защиты и ответственность в правовом аспекте}

Уже упомянутый Федеральный закон <<Об информации, информационных технологиях и о защите информации>>   вводит дисциплинарную, гражданско-правовую, административную или уголовную ответственность за   нарушение интересов и прав лиц, пострадавших от разглашения информации ограниченного доступа или любого другого неправомерного использования данной информации.   Подробности защиты тайны связи в России и мире рассмотрены в пункте 1.1 <<Понятия в правовом аспекте>>
%LITER: http://www.consultant.ru/cons/cgi/online.cgi?req=doc&base=LAW&n=221952&fld=134&dst=100144,0&rnd=0.40247948281489154#07281395619440368
%определяет набор правовых, организационных и технических мер,  целью которых является  защита информации от неправомерного доступа, модификации, блокирования, копирования и распространения. Также вводится ответственность за правонарушения в сфере информационных технологий и защиты информации.
\subsection{Защита переписки при помощи существующего ПО и его анализ}
Далее  рассмотрены существующие способы защиты тайны переписки в интернете с помощью существующего ПО, проведен детальный анализ и выбраны оптимальные средства для конкретных задач, т.е. оптимального баланса простоты использования, доступности и надёжности. Также рассмотрены методы защиты от угроз описанных  в пункте 2.2. 
\\


\textbf{Использование доверенного безопасного  ПО. } В первую очередь, для защиты частной переписки необходимо убедиться в использовании оригинальных программных продуктов, поставляемых надёжными поставщиками. В качестве критериев надёжности можно выбрать:
\begin{itemize}
	\item Популярность. Если продукт находится на рынке достаточно долго, имеет хорошие отзывы и нет известных инцидентов компрометации данного продукта, то такой продукт можно считать <<надёжным>>. 
	\item Получение из оригинальных источников. Используемый продукт необходимо получать только от доверенного поставщика, т.е. ПО должно быть получено от официального дистрибьютора и/или из доверенного источника (официальный сайт, репозиторий).	
	\item Проверка на оригинальность. Для защиты от реверс-инжиниринга и/или внедрения модификации в исполнимый файл или исходный код, если продукт распространяется в таком виде, необходимо использовать валидацию полученного     продукта с помощью хэш-функций, например MDA-5, SHA-256. Такой подход используется как для проприетарного    (пакет Office от Microsoft), так и для open-source ПО (wine, transmission, vim). В противном случае возможно изменение ПО для превращения в кейлоггер или аналогичную программу слежения. \\
\end{itemize}


\textbf{Средства анонимного или шифрованного общения: мессенджеры, ремейлеры, сетевые средства}. Анонимные оверлейные сети -- это сети, работающие поверх уже существующей и работающей сети. Рассмотрим такие примеры таких сетей:

\textbf{Tor, луковая маршрутизация}. Анонимная оверлейная сеть, использующая принцип <<луковой маршрутизации>> -- технология анонимного обмена информацией, использующая многократное шифрование и пересылку шифрованных данных через цепочку частных узлов. Идеи, связанные с ЛМ, впервые появились в конце 90-х годов XX века и активно применялись ВМС США. Основной принцип работы ЛМ и Tor как частного случая: маршрутизатор при старте сессии  передачи выбирает случайное число промежуточных маршрутизаторов, генерирует сообщение для каждого, шифруя их симметричным ключом и указывая для каждого маршрутизатора, какой маршрутизатор будет следующим на пути (структура, аналогичная односвязному списку); для получения симметричного ключа устанавливается начальное соединение с каждым промежуточным маршрутизатором и используется его открытый ключ; таким образом, передаваемые по сети сообщения имеют <<луковую>> структуру, где для получения доступа к содержимому сообщения необходимо поочередно <<снимать слои>> ; каждый маршрутизатор <<снимает один слой>>, получает предназначенные только ему указания маршрутизации (следующий прокси) и шифрованное сообщение, которое необходимо передать далее; последний маршрутизатор <<снимает последний слой>>, отправляет сообщение адресату. Таким образом формируется устойчивая сеть, где каждый прокси передает сообщения в любую сторону, наращивая слои шифрования при передаче ответного сообщения. %LITER http://www.inf.uni-konstanz.de/dbis/teaching/ss03/internet-protocols/download/onion.pdf
%LITER http://cryptome.org/2014/08/onion-routing-security-2000.pdf
%GRAPH https://upload.wikimedia.org/wikipedia/commons/d/dc/Tor-onion-network.png

Преимущества Tor и луковой маршрутизации:высокая степень несвязности сети, прямо зависящая от кол-ва участвующих прокси; возможность работы даже при наличии скомпрометированных узлов, если только вся сеть не стоит из таких узлов; сочетание Tor и других средств шифрования и анонимности позволяет бороться с PRISM. %LITER https://www.pgpru.com/novosti/2013/prismprotivtor


Недостатки: отсутствие защиты от анализа синхронизации в слабонагруженных сетях, отсутствие защиты   от анализа данных, проходящих через выходные узлы, т.к. оператор может получить доступ к данным через сниффинг, если только не используется конечная криптография типа SSL/TSL; уязвимости к атакам MITM,  по времени, по сторонним каналам, глобальному пассивному наблюдению;  %LITER https://www.pgpru.com/faq/anonimnostjobschievoprosy#h37444-7
%LITER https://webcourse.cs.technion.ac.il/236349/Spring2014/ho/WCFiles/2011-01-2.report.pdf
%LITER https://arstechnica.com/information-technology/2013/09/snoops-can-identify-tor-users-given-enough-time-experts-say/
%LITER https://www.freehaven.net/anonbib/topic.html
ошибки  в программной реализации; на последнем узле цепи Tor возможна деанонимизация отправителя или модификация отправляемого сообщения;  при работе с сетью  к сообщениям пользователя может добавляться техническая информация, полностью либо частично раскрывающая отправителя. %LITER https://xakep.ru/2014/10/27/tor-russia/

\textbf{Чесночная маршрутизация, I2P}.I2P --  проект, начатый с целью создания анонимной компьютерной сети, работающей поверх сети интернет. Создатели проекта разработали свободное программное обеспечение (ПО), позволяющее реализовать сеть, работающую поверх сети интернет. Такая сеть является оверлейной, устойчивой к отключению узлов, шифрованной и анонимной к определению IP-адресов. Внутри сети возможно размещение любого сервиса или службы: файлообменник, электронную почту, форум, чат, VoIP) при полном сохранении анонимности сервера. I$ ^{2} $P допускает построение одноранговых сетей типа BitTorrent, Kad, Gnutella. %LITER https://geti2p.net/ru/get-involved/develop/licenses 
Сеть является самоорганизующейся и распределённой, используется модифицированный DHT Kademlia, при этом сеть хранит хешированные адреса узлов сети, зашифрованные AES-протоколом IP-адреса и публичные ключи шифрования, при этом соединения по Network database тоже зашифрованы.	Сеть предоставляет приложениям транспортный механизм для анонимной и защищённой пересылки сообщений друг другу. %LITER Анонимность в сети Интернет // КомпьютерПресс : журнал. — 2010. — № 9.
Благодаря библиотеке Streaming lib реализована  доставка пакетов  в первоначально заданной последовательности без ошибок, потерь и дублирования, что даёт возможность использовать в сети I2P IP-телефонию, интернет-радио, IP-телевидение, видеоконференции и другие потоковые протоколы и сервисы. %LITER Денис Колисниченко. Анонимность и безопасность в Интернете: от "чайника" к пользователю. — БХВ-Петербург, 2011. — С. 44, 46, 47. — 240 с. — ISBN 978-5-9775-0363-1.
Внутри сети  существует автономный каталог сайтов, электронные библиотеки,торрент-трекеры. Также существуют шлюзы для доступа в сеть I2P непосредственно из Интернета, созданные  для пользователей, которые  не могут установить на компьютер программное обеспечение <<Проекта Невидимый Интернет>>. Внутри I2P реализованы механизмы шифрования, P2P (peer to peer )-архитектура, перемены посредников (хопы).

Преимущества: сеть изначально проектировалась с предположением скомпрометированности  всех промежуточных узлов, %LITER Adrian Crenshaw. Darknets and hidden servers: Identifying the true IP/network identity of I2P service hosts // In the Proceedings of Black Hat 2011. — Washington, DC, 2011.
весь трафик шифруется от отправителя до получателя с использованием четырёх уровней шифрования (сквозное, чесночное, туннельное,  шифрование транспортного уровня), добавляется небольшое случайное количество случайных байт; все пакеты зашифровываются на стороне отправителя и расшифровываются только на стороне получателя, при этом никто из промежуточных участников обмена не имеет возможности перехватить расшифрованные данные и никто из участников не знает, кто на самом деле отправитель и кто получатель; сеть устойчива к потере даже значительного (более 50\%) числа узлов и попыткам внешнего анализа.

Недостатки: уязвимость к подмене узлов, при которой  злоумышленник заменяет рабочие узлы на скомпрометированные; перехвата туннелей; атака методом исключения, при которой злоумышленник последовательным перебором может установить, какие маршрутизаторы используются конкретным пользователем; %LITER https://xakep.ru/2014/09/04/i2p-secrets/
Sybil attack, позволяющая без захвата контроля над узлом закрыть доступ узлам сети 	к определённой информации; низкая скорость доступа, для синхронизации с сетью требуется примерно час.

\textbf{JonDo}. ПО, предоставляющее доступ к цепочке прокси-серверов, напоминающее Tor. В отличии  от  Tor, где ноду может создать любой участник, JonDo опирается на помощь отдельных организаций и группировок. К недостаткам относятся все уязвимости Tor, низкая скорость доступа, сильно ограниченное число нод.

\textbf{Ремейлеры} представляют собой серверы, пересылающие сообщения электронной почты по указанному адресу. Делятся на псевдонимные и анонимные. Последние делятся на ремейлеры шифропанков, MixMaster, MixMinion. При использовании псевдо-анонимного ремейлера, его оператор знает адрес электронной почты, который необходим для получения ответа на письмо. Тайна связи полностью зависит от оператора, который может стать жертвой угроз, шантажа или социальной инженерии. Преимуществом псевдо-анонимных ремейлеров является их юзабилити, за которое пользователь расплачивается меньшей защищённостью. Анонимные ремейлеры обеспечивают гораздо более высокую секретность, но при этом они и сложнее в использовании. Их операторы не могут знать, какие данные пересылаются через них, а поэтому нет гарантии своевременной доставки сообщения, которое может и вовсе затеряться.%LITER https://www.pgpru.com/forum/anonimnostjvinternet/kakajatomrachnajaatmosferavokrugremejjlerovremailers
В обмен на высокое время ожидания анонимные ремейлеры достаточно надёжно скрывают от посторонних глаз реальный адрес и содержимое сообщения. 
\begin{itemize}
	\item Ремейлеры шифропанков удаляют из полученных писем всю информацию, которая может быть использована для идентификации отправителя, и пересылают письмо на указанный адрес. Чаще всего используется PGP-шифрование, возможно создание цепочки таких ремейлеров.
	\item MixMaster. Требуют установки и более совершенны, чем прошлый тип, т.к отправляемые сообщения всегда константного размера, что делает невозможной отслежу по размерам.
	\item MixMinion.  Стандарт реализации третьего типа протокола анонимной пересылки электронной почты, может отсылать и принимать анонимные сообщения электронной почты, основан на пересылаемых защищённых одноразовых блоках. %LITER https://gnunet.org/sites/default/files/minion-design.pdf
\end{itemize}
Ремейлеры также имеют ряд уязвимостей и недостатков: тэговая атака, атака на выходные узлы, DDoS, путь доставки сообщений не всегда является оптимальным.

\textbf{Стеганография}. Способ тайной передачи информации путем сохранения в тайне самого факта передачи информации. В отличие от криптографии, которая скрывает содержимое тайного сообщения, стеганография скрывает сам факт его существования. Как правило, сообщение будет выглядеть как что-либо иное, например, как изображение, статья, список покупок, письмо или судоку. Стеганографию обычно используют совместно с методами криптографии, таким образом, дополняя её. Термин введен  в конце XV века монахом Иоганном Тритемием в трактате <<Steganographia>>, зашифрованном под магическую книгу. Криптография защищает содержание сообщения,  стеганография --  сам факт наличия каких-либо скрытых посланий.  Существует множество аналоговых методов стеганографии, однако в контексте данной работы рассматриваются только цифровые методы.  \textbf{Цифровая стеганография} — направление  стеганографии, основанное на сокрытии и/или внедрении дополнительной информации в цифровые объекты, вызывая при этом   искажения этих объектов. Как правило, данные объекты являются мультимедиа-объектами (изображения, видео, аудио, текстуры 3D-объектов) и внесение искажений, находящихся вне порога чувствительности среднестатистического человека, не приводит к значительным изменениям этих объектов. Также, в оцифрованных объектах, первоначально имеющих аналоговую природу, всегда присутствует шум квантования. Существующие алгоритмы цифровой стеганографии:
\begin{itemize}
	\item Сетевая стеганография, основанная на использовании особенностей работы сетевых протоколов передачи данных, когда части отдельных пакетов заменяются битами секретной информации. %LITER http://ijigroup.com/index.php?id=81
	\item Встраивание в изображение, при этом алгоритм работает непосредственно с цифровым сигналом (LSB), накладывает (fusion) изображение поверх существующего (цифровые водяные знаки), использование особенностей файлов (хранение в метаданных или неиспользуемых полях).
	\item Эхо-методы аудио-стеганографии, использующие неравномерные промежутки между эхо-сигналами для кодирования последовательности значений. 
\end{itemize}
Необходимо помнить, что если алгоритм  стеганографии или используемое ПО станут известны аналитику, анализ может быть проведен за адекватное время, поэтому перед кодировкой сообщений их необходимо шифровать с помощью криптографии. %LITER https://www.alternet.org/story/11986/confounding_carnivore%3A_how_to_protect_your_online_privacy


В целом, стеганография является качественным методом для передачи небольших (по сравнению с файлом, в который происходит встраивание) текстовых сообщений или мультимедиа объектов. Однако, стеганография не имеет широкой популярности и имеет уязвимости: атака на основании известного заполненного контейнера,  на основании известного встроенного сообщения,  на основании выбранного встроенного сообщения, на основании известного пустого контейнера, на основании выбранного пустого контейнера, на основании известной математической модели контейнера или его части.
%GRAPH steganography

\textbf{Мессенджеры}. Наиболее классическая и очевидная область применения существующих наработок в области защиты тайны связи, т.к. содержат преимущественно частную переписку, подвергающуюся анализу со стороны злоумышленников и государств.  Ниже рассмотрены самые популярные мировые мессенджеры (РФ в целом следует европейским трендам). %LITER https://superfamilyprotector.com/blog/en/most-popular-messaging-apps-2018/
%LITER https://www.similarweb.com/blog/worldwide-messaging-apps
%LITER https://akket.com/raznoe/73783-top-10-samyh-populyarnyh-messendzherov-v-rossii-kotorymi-polzuyutsya-vse-rossiyane.html
%LITER https://www.rbc.ru/technology_and_media/18/01/2016/569cddd29a794722c534df2c

\textbf{Viber}. Первый по популярности мессенджер  в РФ. В 2016 году Viber получил сквозное (end-to-end ) шифрование, однако подробности работы, например, симметричность или асимметричность, метод распределения ключей.  При этом через шифрование проходят как текстовые, так и мультимедийные объекты, пересылаемые в диалогах.ля этого каждый из клиентов использует открытый и закрытый ключ. Их генерирование осуществляется автоматически в момент установки программы на устройствах обеих сторон, что является потенциальной уязвимостью, так как ключи не уникальны для каждого диалога.%LITER http://allmessengers.ru/viber/shifrovanie
%LITER https://www.viber.com/ru/security/
Существует мнение, что <<поскольку и алгоритмы шифрования, и ключи шифрования, и уже расшифрованные сообщения находятся внутри софта мессенджера на конечном устройстве, доступ к любой информации у владельца мессенджера может быть>> и Viber не является в этом отношении исключением, кроме того <<Viber компрометируют себя функцией создания копий истории переписки>>. %LITER https://www.kommersant.ru/doc/3379658
Более того, имел место инцидент со взломом одного из вспомогательных сервисов Viber Support в 2013 году группировкой <<Syrian Electronic Army>>. %LITER https://www.engadget.com/2013/07/23/viber-support-page-hacked-by-syrian-electronic-army/ 
В 2015, согласно Закону << О персональных данных>>, требующего хранения персональных данных россиян на территории РФ, Viber принял решения о переносе номеров телефонов и никнеймов на территорию РФ, предоставив доступ правоохранительным органам. Были высказаны опасения, что дата-центры могут быть атакованы злоумышленниками на территории РФ, что потенциально ведет к значительной утечке данных. %LITER http://www.the-village.ru/village/business/news/224095-viber-personal-data
\newpage
	\newpage
\parindent=1cm %красная строка
\addcontentsline{toc}{section}{Заключение} %Убираем номер , даём имя в оглавлении 
\section*{Заключение} %сам текст заголовка 

Разработанный и протестированный алгоритм имеет как преимущества, так и недостатки, рассмотрим их подробнее.


Преимущества: \\
\begin{itemize}
	\item Полная симуляция одновременно происходящих параллельных процессов;
	\item Минимальное, по сравнению с другими рассмотренными алгоритмами, число итераций;
	\item Возможность экспортировать данные в табличные процессоры для пост-обработки;
	\item Низкие затраты по памяти -- запущенное приложение при 10 активных ракетах-целях, 30 активных  ракетах-перехватчиках и 100 ракетах-перехватчиков на складе потребляет около 500 Мб ОЗУ;
	\item Высокие возможности распараллеливания алгоритма на критических участках -- потенциальный источник еще большей оптимизации исполнения алгоритма.
\end{itemize}

Недостатки: \\
\begin{itemize}
	\item Не самая высокая скорость сходимости среди рассмотренных;
	\item Симуляция параллельных событий требует процессора со значительным числом ядер и логических потоков, которые часто отсутствуют на встраиваемых системах, реально используемых в подобной технике. Так, описанное выше кол-во объектов создает около 50 процессов и более 100 потоков;
	\item Алгоритм рассматривает упрощенную схему полёта ракеты-цели и не учитывает возможных положений, допускающих оптимизацию процесса перехвата.
\end{itemize}

Были решены следующие задачи:

\begin{itemize}
	\item Описана СПРО для которой  построена модель и агенты, участвующие в ней;
	\item Описана система принятия решений и накладываемые на нее ограничения, алгоритм принятия решений;
	\item алгоритм принятия решений был реализован в виде ПО, протестирован и сравнён с другими подобными алгоритмами;
	\item были решены различные задачи, возникающие в области многоточечного программирования;
	\item были сделаны выводы о потенциале и возможностях развития развития данного алгоритма.
\end{itemize}

К сожалению, не удалось реализовать дополнительную цель и построить полноценный интерфейс приложения, позволяющий ввод-вывод данных и, самое важное, демонстрацию работы алгоритма посредством показа назначения ракет-перехватчиков на цели и непосредственно перехватом.

Несмотря на неидеальные параметры работы алгоритма, он всё же демонстрирует высокую скорость исполнения. Отметим, что как сам алгоритм АРЯПОСОО, так и модель, имитирующая СПРО, допускают в будущем модификации как расширяющие функционал СПРО, так и модифицирующие непосредственно работу самого алгоритма АРЯПОСОО.


	
\addcontentsline{toc}{section}{Список использованной литературы} %список литературы
	\newpage
	
\end{document}
