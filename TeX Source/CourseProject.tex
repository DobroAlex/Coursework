
\documentclass[a4paper,14pt,russian]{extreport}	%A4 бумага, 12 кегль, русский язык 
\usepackage{extsizes}
\usepackage[onehalfspacing]{setspace} % поулторный интервал %https://proft.me/2013/06/9/latex-ukazanie-mezhstrochnogo-intervala

\usepackage{cmap} % для кодировки шрифтов в pdf
\usepackage[T2A]{fontenc}
%\usepackage{pscyr}
%\usepackage{graphicx} % для вставки картинок
\usepackage{mathptmx} %поддержка textbf
\usepackage{makecell}
\usepackage{textcomp}
\usepackage{multirow} % улучшенное форматирование таблиц
\usepackage{ulem} % подчеркивания

%полужирный шрифт http://tostudents.ru/2009/12/08/smena-shriftov-v-latex-tekst-i-formuly/
\renewcommand{\rmdefault}{ftm} % Times New Roman
\usepackage[utf8]{inputenc}%включаем свою кодировку: koi8-r или utf8 в UNIX, cp1251 в Windows
%\usepackage[]{babel}	%больше поддержки русского языка 
\usepackage[english,russian, russianb]{babel}%используем русский и английский языки с переносами
\usepackage{amssymb,amsfonts,amsmath,mathtext,cite,enumerate,float} %подключаем нужные пакеты расширений
\usepackage[dvips]{graphicx} %хотим вставлять в диплом рисунки?
\usepackage{cmap} % Улучшенный поиск русских слов в полученном pdf-файле
%\graphicspath{{images/}}%путь к рисункам
\usepackage{fancyhdr}%оформление нумерации 
\usepackage{tableof} %поддержка табличек
\makeatletter
\renewcommand{\@biblabel}[1]{#1.} % Заменяем библиографию с квадратных скобок на точку:
\makeatother

\usepackage{geometry} % Меняем поля страницы
%QUEST: 3cm или 2 ? Ес\usepackage{cmap} % Улучшенный поиск русских слов в полученном pdf-файлели 3, придется менять форматирование заголовка 
\geometry{left=3cm}% левое поле
\geometry{right=2cm}% правое поле
\geometry{top=2cm}% верхнее поле
\geometry{bottom=2cm}% нижнее поле


\renewcommand{\theenumi}{\arabic{enumi}}% Меняем везде перечисления на цифра.цифра
\renewcommand{\labelenumi}{\arabic{enumi}}% Меняем везде перечисления на цифра.цифра
\renewcommand{\theenumii}{.\arabic{enumii}}% Меняем везде перечисления на цифра.цифра
\renewcommand{\labelenumii}{\arabic{enumi}.\arabic{enumii}.}% Меняем везде перечисления на цифра.цифра
\renewcommand{\theenumiii}{.\arabic{enumiii}}% Меняем везде перечисления на цифра.цифра
\renewcommand{\labelenumiii}{\arabic{enumi}.\arabic{enumii}.\arabic{enumiii}.}% Меняем везде перечисления на цифра.цифра

\renewcommand{\rmdefault}{ftm}
%NB: три команды ниже переопределяют некотрые шрифты  и дают поддержку жирного и прочиах текстов https://www.linux.org.ru/forum/general/4219163
\renewcommand{\rmdefault}{cmr} % Шрифт с засечками
\renewcommand{\sfdefault}{cmss} % Шрифт без засечек
\renewcommand{\ttdefault}{cmtt} % Моноширинный шрифт
\renewcommand*\thesection{\arabic{section}}
    
\begin{document}
	%\pagestyle{empty} % нумерация выкл.
	    \begin{titlepage}
    \newpage
	\pagestyle{empty} % нумерация выкл.
    \begin{center}
    \normalsize МИНИСТЕРСТВО ОБРАЗОВАНИЯ И НАУКИ РОССИЙСКОЙ ФЕДЕРАЦИИ\\ 
    \small  {Федеральное государственное автономное образовательное учреждение высшего образования} 
    \large \textbf{<<Крымский  федеральный  университет имени В. И. Вернадского>>} \\  \vspace{2mm}
    Таврическая академия (структурное подразделение ) \\
    \vspace{2mm}
    Факультет математики и информатики \\
    \vspace{2mm}
    Кафедра прикладной математики 
    \end{center}
    \vspace{3em}

    \begin{center}
	\normalsize Консманов Алексей Витальевич \\
    \LARGE \textbf{Сохранение тайны связи в условиях новых цифровых угроз} \\
    \vspace{1em}
    \normalsize Курсовая работа 
    \end{center}

    \vspace{1em}
    
    \begin{center}
    	\begin{tabbing}	%http://www.intuit.ru/studies/courses/1137/137/lecture/3835%3Fpage%3D5
    		\hspace{3cm}Обучающегося \hspace{3cm} \textbf{3 курса}\\ %Быдлокод?
    		\hspace{3cm}Направления подготовки \hspace{7mm}  \textbf{01.03.04}\\
    		\hspace{3cm}Форма обучения \hspace{26mm} \textbf{очная}
    	\end{tabbing}
    
	\vspace {3em}
    \flushleft Научный руководитель \hspace{20mm}  старший преподаватель 
    
    \hspace{75mm}кафедры прикладной математики  
    
    
    \hspace{75mm}В. А. Лушников
	\end{center}
    \vspace{\fill}

    \begin{center}
    Симферополь 2018
    \end{center}

    \end{titlepage}
% это титульный лист

	\tableofcontents % это оглавление, которое генерируется автоматически
	\thispagestyle{empty}%отключает нумерование страниц до введения включительно 
	%\addcontentsline{toc}{section}{Введение}% будет костыльно выглядеть
	\newpage
\parindent=1cm %красная строка? 
\begin{center}
	\addcontentsline{toc}{section}{Введение} %Убираем номер , даём имя в оглавлении 
	\section*{Введение} %сам текст заголовка 
	\pagestyle{plain} % нумерация выкл.
	\setcounter{page}{3} % начать нумерацию с номера тр
\end{center}

Актуальность работы  связана с возросшим числом новых угроз в области защиты личных данных, участившимися атаками частных лиц, группировок и специальных ведомств иностранных государств против частных лиц с целью получения частной информации, анализа полученных личных данных   и использования для шантажа атакуемых лиц, продажи или другого выгодного обмена, а также  в иных противозаконных целях. 

Целью данной работы является %EDIT: сделать это в виде списка с пунктами?
 анализ новых цифровых угроз, возникших в последнее десятилетие в связи с бурным развитием информационных технологий, за которым не последовал соразмерный рост знаний пользователей цифровых систем, используемые кибер-преступниками методы анализа и атаки на частные данные, правовой аспект защиты личной переписки, способы борьбы с угрозами  в рамках существующего программного обеспечения, %EDIT: заменить на ПО и создать список используемых сокращений до введения?
 сравнительный анализ существующих продуктов, разработка и реализация собственных алгоритмов для сохранения тайны личной переписки. 	 %введение
	\parindent=1cm %красная строка? 
\begin{center}
		
		\section{Понятия <<тайна связи>> и <<личная переписка>> в правом и информационном аспектах}
		
\end{center}

\subsection{Понятия в правовом аспекте} 

Так как все пользователь информационных систем являются в первую очередь гражданами правовых государств и объектами и субъектами права, рассмотрение основных понятий начнём с правого аспекта вопроса.

Согласно статье 63 федерального закона <<О связи>>: %LITER: http://home.garant.ru/#/document/186117/paragraph/645:1
<<На территории Российской Федерации гарантируется тайна переписки, телефонных переговоров, почтовых отправлений, телеграфных и иных сообщений, передаваемых по сетям электросвязи и сетям почтовой связи.>> Исходя из данного закона, в дальнейшем под <<\textbf{тайной связи}>>  будет подразумеваться совокупность тайны переписки, телефонных разговоров, почтовых отправлений, телеграфных и иных сообщений, передаваемых по сетям электросвязи и сетям почтовой связи. Также, в дальнейшем к сетям электросвязи будем относить и Интернет. 


 
\newpage % следующая глава с новой строки
	\parindent=1cm %красная строка
\begin{center}
		
		\section{Защита личной переписки}
		
\end{center}

Принимая во внимание большое число угроз, рассмотрим существующие правовые и фактические способы обеспечения секретности тайны связи.
\subsection{Способы защиты и ответственность в правовом аспекте}

Уже упомянутый Федеральный закон <<Об информации, информационных технологиях и о защите информации>>   вводит дисциплинарную, гражданско-правовую, административную или уголовную ответственность за   нарушение интересов и прав лиц, пострадавших от разглашения информации ограниченного доступа или любого другого неправомерного использования данной информации.   Подробности защиты тайны связи в России и мире рассмотрены в пункте 1.1 <<Понятия в правовом аспекте>>
%LITER: http://www.consultant.ru/cons/cgi/online.cgi?req=doc&base=LAW&n=221952&fld=134&dst=100144,0&rnd=0.40247948281489154#07281395619440368
%определяет набор правовых, организационных и технических мер,  целью которых является  защита информации от неправомерного доступа, модификации, блокирования, копирования и распространения. Также вводится ответственность за правонарушения в сфере информационных технологий и защиты информации.
\subsection{Защита переписки при помощи существующего ПО и его анализ}
Далее  рассмотрены существующие способы защиты тайны переписки в интернете с помощью существующего ПО, проведен детальный анализ и выбраны оптимальные средства для конкретных задач, т.е. оптимального баланса простоты использования, доступности и надёжности. Также рассмотрены методы защиты от угроз описанных  в пункте 2.2. 
\\


\textbf{Использование доверенного безопасного  ПО. } В первую очередь, для защиты частной переписки необходимо убедиться в использовании оригинальных программных продуктов, поставляемых надёжными поставщиками. В качестве критериев надёжности можно выбрать:
\begin{itemize}
	\item Популярность. Если продукт находится на рынке достаточно долго, имеет хорошие отзывы и нет известных инцидентов компрометации данного продукта, то такой продукт можно считать <<надёжным>>. 
	\item Получение из оригинальных источников. Используемый продукт необходимо получать только от доверенного поставщика, т.е. ПО должно быть получено от официального дистрибьютора и/или из доверенного источника (официальный сайт, репозиторий).	
	\item Проверка на оригинальность. Для защиты от реверс-инжиниринга и/или внедрения модификации в исполнимый файл или исходный код, если продукт распространяется в таком виде, необходимо использовать валидацию полученного     продукта с помощью хэш-функций, например MDA-5, SHA-256. Такой подход используется как для проприетарного    (пакет Office от Microsoft), так и для open-source ПО (wine, transmission, vim). В противном случае возможно изменение ПО для превращения в кейлоггер или аналогичную программу слежения. \\
\end{itemize}


\textbf{Средства анонимного или шифрованного общения: мессенджеры, ремейлеры, сетевые средства}. Анонимные оверлейные сети -- это сети, работающие поверх уже существующей и работающей сети. Рассмотрим такие примеры таких сетей:

\textbf{Tor, луковая маршрутизация}. Анонимная оверлейная сеть, использующая принцип <<луковой маршрутизации>> -- технология анонимного обмена информацией, использующая многократное шифрование и пересылку шифрованных данных через цепочку частных узлов. Идеи, связанные с ЛМ, впервые появились в конце 90-х годов XX века и активно применялись ВМС США. Основной принцип работы ЛМ и Tor как частного случая: маршрутизатор при старте сессии  передачи выбирает случайное число промежуточных маршрутизаторов, генерирует сообщение для каждого, шифруя их симметричным ключом и указывая для каждого маршрутизатора, какой маршрутизатор будет следующим на пути (структура, аналогичная односвязному списку); для получения симметричного ключа устанавливается начальное соединение с каждым промежуточным маршрутизатором и используется его открытый ключ; таким образом, передаваемые по сети сообщения имеют <<луковую>> структуру, где для получения доступа к содержимому сообщения необходимо поочередно <<снимать слои>> ; каждый маршрутизатор <<снимает один слой>>, получает предназначенные только ему указания маршрутизации (следующий прокси) и шифрованное сообщение, которое необходимо передать далее; последний маршрутизатор <<снимает последний слой>>, отправляет сообщение адресату. Таким образом формируется устойчивая сеть, где каждый прокси передает сообщения в любую сторону, наращивая слои шифрования при передаче ответного сообщения. %LITER http://www.inf.uni-konstanz.de/dbis/teaching/ss03/internet-protocols/download/onion.pdf
%LITER http://cryptome.org/2014/08/onion-routing-security-2000.pdf
%GRAPH https://upload.wikimedia.org/wikipedia/commons/d/dc/Tor-onion-network.png

Преимущества Tor и луковой маршрутизации:высокая степень несвязности сети, прямо зависящая от кол-ва участвующих прокси; возможность работы даже при наличии скомпрометированных узлов, если только вся сеть не стоит из таких узлов; сочетание Tor и других средств шифрования и анонимности позволяет бороться с PRISM. %LITER https://www.pgpru.com/novosti/2013/prismprotivtor


Недостатки: отсутствие защиты от анализа синхронизации в слабонагруженных сетях, отсутствие защиты   от анализа данных, проходящих через выходные узлы, т.к. оператор может получить доступ к данным через сниффинг, если только не используется конечная криптография типа SSL/TSL; уязвимости к атакам MITM,  по времени, по сторонним каналам, глобальному пассивному наблюдению;  %LITER https://www.pgpru.com/faq/anonimnostjobschievoprosy#h37444-7
%LITER https://webcourse.cs.technion.ac.il/236349/Spring2014/ho/WCFiles/2011-01-2.report.pdf
%LITER https://arstechnica.com/information-technology/2013/09/snoops-can-identify-tor-users-given-enough-time-experts-say/
%LITER https://www.freehaven.net/anonbib/topic.html
ошибки  в программной реализации; на последнем узле цепи Tor возможна деанонимизация отправителя или модификация отправляемого сообщения;  при работе с сетью  к сообщениям пользователя может добавляться техническая информация, полностью либо частично раскрывающая отправителя. %LITER https://xakep.ru/2014/10/27/tor-russia/

\textbf{Чесночная маршрутизация, I2P}.I2P --  проект, начатый с целью создания анонимной компьютерной сети, работающей поверх сети интернет. Создатели проекта разработали свободное программное обеспечение (ПО), позволяющее реализовать сеть, работающую поверх сети интернет. Такая сеть является оверлейной, устойчивой к отключению узлов, шифрованной и анонимной к определению IP-адресов. Внутри сети возможно размещение любого сервиса или службы: файлообменник, электронную почту, форум, чат, VoIP) при полном сохранении анонимности сервера. I$ ^{2} $P допускает построение одноранговых сетей типа BitTorrent, Kad, Gnutella. %LITER https://geti2p.net/ru/get-involved/develop/licenses 
Сеть является самоорганизующейся и распределённой, используется модифицированный DHT Kademlia, при этом сеть хранит хешированные адреса узлов сети, зашифрованные AES-протоколом IP-адреса и публичные ключи шифрования, при этом соединения по Network database тоже зашифрованы.	Сеть предоставляет приложениям транспортный механизм для анонимной и защищённой пересылки сообщений друг другу. %LITER Анонимность в сети Интернет // КомпьютерПресс : журнал. — 2010. — № 9.
Благодаря библиотеке Streaming lib реализована  доставка пакетов  в первоначально заданной последовательности без ошибок, потерь и дублирования, что даёт возможность использовать в сети I2P IP-телефонию, интернет-радио, IP-телевидение, видеоконференции и другие потоковые протоколы и сервисы. %LITER Денис Колисниченко. Анонимность и безопасность в Интернете: от "чайника" к пользователю. — БХВ-Петербург, 2011. — С. 44, 46, 47. — 240 с. — ISBN 978-5-9775-0363-1.
Внутри сети  существует автономный каталог сайтов, электронные библиотеки,торрент-трекеры. Также существуют шлюзы для доступа в сеть I2P непосредственно из Интернета, созданные  для пользователей, которые  не могут установить на компьютер программное обеспечение <<Проекта Невидимый Интернет>>. Внутри I2P реализованы механизмы шифрования, P2P (peer to peer )-архитектура, перемены посредников (хопы).

Преимущества: сеть изначально проектировалась с предположением скомпрометированности  всех промежуточных узлов, %LITER Adrian Crenshaw. Darknets and hidden servers: Identifying the true IP/network identity of I2P service hosts // In the Proceedings of Black Hat 2011. — Washington, DC, 2011.
весь трафик шифруется от отправителя до получателя с использованием четырёх уровней шифрования (сквозное, чесночное, туннельное,  шифрование транспортного уровня), добавляется небольшое случайное количество случайных байт; все пакеты зашифровываются на стороне отправителя и расшифровываются только на стороне получателя, при этом никто из промежуточных участников обмена не имеет возможности перехватить расшифрованные данные и никто из участников не знает, кто на самом деле отправитель и кто получатель; сеть устойчива к потере даже значительного (более 50\%) числа узлов и попыткам внешнего анализа.

Недостатки: уязвимость к подмене узлов, при которой  злоумышленник заменяет рабочие узлы на скомпрометированные; перехвата туннелей; атака методом исключения, при которой злоумышленник последовательным перебором может установить, какие маршрутизаторы используются конкретным пользователем; %LITER https://xakep.ru/2014/09/04/i2p-secrets/
Sybil attack, позволяющая без захвата контроля над узлом закрыть доступ узлам сети 	к определённой информации; низкая скорость доступа, для синхронизации с сетью требуется примерно час.

\textbf{JonDo}. ПО, предоставляющее доступ к цепочке прокси-серверов, напоминающее Tor. В отличии  от  Tor, где ноду может создать любой участник, JonDo опирается на помощь отдельных организаций и группировок. К недостаткам относятся все уязвимости Tor, низкая скорость доступа, сильно ограниченное число нод.

\textbf{Ремейлеры} представляют собой серверы, пересылающие сообщения электронной почты по указанному адресу. Делятся на псевдонимные и анонимные. Последние делятся на ремейлеры шифропанков, MixMaster, MixMinion. При использовании псевдо-анонимного ремейлера, его оператор знает адрес электронной почты, который необходим для получения ответа на письмо. Тайна связи полностью зависит от оператора, который может стать жертвой угроз, шантажа или социальной инженерии. Преимуществом псевдо-анонимных ремейлеров является их юзабилити, за которое пользователь расплачивается меньшей защищённостью. Анонимные ремейлеры обеспечивают гораздо более высокую секретность, но при этом они и сложнее в использовании. Их операторы не могут знать, какие данные пересылаются через них, а поэтому нет гарантии своевременной доставки сообщения, которое может и вовсе затеряться.%LITER https://www.pgpru.com/forum/anonimnostjvinternet/kakajatomrachnajaatmosferavokrugremejjlerovremailers
В обмен на высокое время ожидания анонимные ремейлеры достаточно надёжно скрывают от посторонних глаз реальный адрес и содержимое сообщения. 
\begin{itemize}
	\item Ремейлеры шифропанков удаляют из полученных писем всю информацию, которая может быть использована для идентификации отправителя, и пересылают письмо на указанный адрес. Чаще всего используется PGP-шифрование, возможно создание цепочки таких ремейлеров.
	\item MixMaster. Требуют установки и более совершенны, чем прошлый тип, т.к отправляемые сообщения всегда константного размера, что делает невозможной отслежу по размерам.
	\item MixMinion.  Стандарт реализации третьего типа протокола анонимной пересылки электронной почты, может отсылать и принимать анонимные сообщения электронной почты, основан на пересылаемых защищённых одноразовых блоках. %LITER https://gnunet.org/sites/default/files/minion-design.pdf
\end{itemize}
Ремейлеры также имеют ряд уязвимостей и недостатков: тэговая атака, атака на выходные узлы, DDoS, путь доставки сообщений не всегда является оптимальным.

\textbf{Стеганография}. Способ тайной передачи информации путем сохранения в тайне самого факта передачи информации. В отличие от криптографии, которая скрывает содержимое тайного сообщения, стеганография скрывает сам факт его существования. Как правило, сообщение будет выглядеть как что-либо иное, например, как изображение, статья, список покупок, письмо или судоку. Стеганографию обычно используют совместно с методами криптографии, таким образом, дополняя её. Термин введен  в конце XV века монахом Иоганном Тритемием в трактате <<Steganographia>>, зашифрованном под магическую книгу. Криптография защищает содержание сообщения,  стеганография --  сам факт наличия каких-либо скрытых посланий.  Существует множество аналоговых методов стеганографии, однако в контексте данной работы рассматриваются только цифровые методы.  \textbf{Цифровая стеганография} — направление  стеганографии, основанное на сокрытии и/или внедрении дополнительной информации в цифровые объекты, вызывая при этом   искажения этих объектов. Как правило, данные объекты являются мультимедиа-объектами (изображения, видео, аудио, текстуры 3D-объектов) и внесение искажений, находящихся вне порога чувствительности среднестатистического человека, не приводит к значительным изменениям этих объектов. Также, в оцифрованных объектах, первоначально имеющих аналоговую природу, всегда присутствует шум квантования. Существующие алгоритмы цифровой стеганографии:
\begin{itemize}
	\item Сетевая стеганография, основанная на использовании особенностей работы сетевых протоколов передачи данных, когда части отдельных пакетов заменяются битами секретной информации. %LITER http://ijigroup.com/index.php?id=81
	\item Встраивание в изображение, при этом алгоритм работает непосредственно с цифровым сигналом (LSB), накладывает (fusion) изображение поверх существующего (цифровые водяные знаки), использование особенностей файлов (хранение в метаданных или неиспользуемых полях).
	\item Эхо-методы аудио-стеганографии, использующие неравномерные промежутки между эхо-сигналами для кодирования последовательности значений. 
\end{itemize}
Необходимо помнить, что если алгоритм  стеганографии или используемое ПО станут известны аналитику, анализ может быть проведен за адекватное время, поэтому перед кодировкой сообщений их необходимо шифровать с помощью криптографии. %LITER https://www.alternet.org/story/11986/confounding_carnivore%3A_how_to_protect_your_online_privacy


В целом, стеганография является качественным методом для передачи небольших (по сравнению с файлом, в который происходит встраивание) текстовых сообщений или мультимедиа объектов. Однако, стеганография не имеет широкой популярности и имеет уязвимости: атака на основании известного заполненного контейнера,  на основании известного встроенного сообщения,  на основании выбранного встроенного сообщения, на основании известного пустого контейнера, на основании выбранного пустого контейнера, на основании известной математической модели контейнера или его части.
%GRAPH steganography

\textbf{Мессенджеры}. Наиболее классическая и очевидная область применения существующих наработок в области защиты тайны связи, т.к. содержат преимущественно частную переписку, подвергающуюся анализу со стороны злоумышленников и государств.  Ниже рассмотрены самые популярные мировые мессенджеры (РФ в целом следует европейским трендам). %LITER https://superfamilyprotector.com/blog/en/most-popular-messaging-apps-2018/
%LITER https://www.similarweb.com/blog/worldwide-messaging-apps
%LITER https://akket.com/raznoe/73783-top-10-samyh-populyarnyh-messendzherov-v-rossii-kotorymi-polzuyutsya-vse-rossiyane.html
%LITER https://www.rbc.ru/technology_and_media/18/01/2016/569cddd29a794722c534df2c

\textbf{Viber}. Первый по популярности мессенджер  в РФ. В 2016 году Viber получил сквозное (end-to-end ) шифрование, однако подробности работы, например, симметричность или асимметричность, метод распределения ключей.  При этом через шифрование проходят как текстовые, так и мультимедийные объекты, пересылаемые в диалогах.ля этого каждый из клиентов использует открытый и закрытый ключ. Их генерирование осуществляется автоматически в момент установки программы на устройствах обеих сторон, что является потенциальной уязвимостью, так как ключи не уникальны для каждого диалога.%LITER http://allmessengers.ru/viber/shifrovanie
%LITER https://www.viber.com/ru/security/
Существует мнение, что <<поскольку и алгоритмы шифрования, и ключи шифрования, и уже расшифрованные сообщения находятся внутри софта мессенджера на конечном устройстве, доступ к любой информации у владельца мессенджера может быть>> и Viber не является в этом отношении исключением, кроме того <<Viber компрометируют себя функцией создания копий истории переписки>>. %LITER https://www.kommersant.ru/doc/3379658
Более того, имел место инцидент со взломом одного из вспомогательных сервисов Viber Support в 2013 году группировкой <<Syrian Electronic Army>>. %LITER https://www.engadget.com/2013/07/23/viber-support-page-hacked-by-syrian-electronic-army/ 
В 2015, согласно Закону << О персональных данных>>, требующего хранения персональных данных россиян на территории РФ, Viber принял решения о переносе номеров телефонов и никнеймов на территорию РФ, предоставив доступ правоохранительным органам. Были высказаны опасения, что дата-центры могут быть атакованы злоумышленниками на территории РФ, что потенциально ведет к значительной утечке данных. %LITER http://www.the-village.ru/village/business/news/224095-viber-personal-data
\newpage
	
\addcontentsline{toc}{section}{Список использованной литературы} %список литературы
	\newpage
	
\end{document}
